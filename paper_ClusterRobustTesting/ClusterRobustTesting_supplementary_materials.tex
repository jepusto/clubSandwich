\documentclass{article}\usepackage[]{graphicx}\usepackage[]{color}
%% maxwidth is the original width if it is less than linewidth
%% otherwise use linewidth (to make sure the graphics do not exceed the margin)
\makeatletter
\def\maxwidth{ %
  \ifdim\Gin@nat@width>\linewidth
    \linewidth
  \else
    \Gin@nat@width
  \fi
}
\makeatother

\definecolor{fgcolor}{rgb}{0.345, 0.345, 0.345}
\newcommand{\hlnum}[1]{\textcolor[rgb]{0.686,0.059,0.569}{#1}}%
\newcommand{\hlstr}[1]{\textcolor[rgb]{0.192,0.494,0.8}{#1}}%
\newcommand{\hlcom}[1]{\textcolor[rgb]{0.678,0.584,0.686}{\textit{#1}}}%
\newcommand{\hlopt}[1]{\textcolor[rgb]{0,0,0}{#1}}%
\newcommand{\hlstd}[1]{\textcolor[rgb]{0.345,0.345,0.345}{#1}}%
\newcommand{\hlkwa}[1]{\textcolor[rgb]{0.161,0.373,0.58}{\textbf{#1}}}%
\newcommand{\hlkwb}[1]{\textcolor[rgb]{0.69,0.353,0.396}{#1}}%
\newcommand{\hlkwc}[1]{\textcolor[rgb]{0.333,0.667,0.333}{#1}}%
\newcommand{\hlkwd}[1]{\textcolor[rgb]{0.737,0.353,0.396}{\textbf{#1}}}%

\usepackage{framed}
\makeatletter
\newenvironment{kframe}{%
 \def\at@end@of@kframe{}%
 \ifinner\ifhmode%
  \def\at@end@of@kframe{\end{minipage}}%
  \begin{minipage}{\columnwidth}%
 \fi\fi%
 \def\FrameCommand##1{\hskip\@totalleftmargin \hskip-\fboxsep
 \colorbox{shadecolor}{##1}\hskip-\fboxsep
     % There is no \\@totalrightmargin, so:
     \hskip-\linewidth \hskip-\@totalleftmargin \hskip\columnwidth}%
 \MakeFramed {\advance\hsize-\width
   \@totalleftmargin\z@ \linewidth\hsize
   \@setminipage}}%
 {\par\unskip\endMakeFramed%
 \at@end@of@kframe}
\makeatother

\definecolor{shadecolor}{rgb}{.97, .97, .97}
\definecolor{messagecolor}{rgb}{0, 0, 0}
\definecolor{warningcolor}{rgb}{1, 0, 1}
\definecolor{errorcolor}{rgb}{1, 0, 0}
\newenvironment{knitrout}{}{} % an empty environment to be redefined in TeX

\usepackage{alltt}
\usepackage[margin=1in]{geometry}
\usepackage{amsmath}
\usepackage{amsthm}
\usepackage{graphicx,psfrag,epsf}
\renewcommand{\thesection}{S\arabic{section}}
\renewcommand{\thefigure}{S\arabic{figure}}
\RequirePackage[natbibapa]{apacite}
\usepackage{caption}
\usepackage{multirow}
\usepackage{float}    % for fig.pos='H'
\usepackage{rotfloat} % for sidewaysfigure
\usepackage{pdflscape}

\newcommand{\Prob}{\text{Pr}}
\newcommand{\E}{\text{E}}
\newcommand{\Cov}{\text{Cov}}
\newcommand{\corr}{\text{corr}}
\newcommand{\Var}{\text{Var}}
\newcommand{\iid}{\stackrel{\text{iid}}{\sim}}
\newcommand{\tr}{\text{tr}}
\newcommand{\bm}{\mathbf}
\newcommand{\bs}{\boldsymbol}

\title{Supplementary materials for \textit{Small sample methods for cluster-robust variance estimation and hypothesis testing in fixed effects models}}
%\author{James E. Pustejovsky \& Elizabeth Tipton}
\IfFileExists{upquote.sty}{\usepackage{upquote}}{}
\begin{document}
\maketitle

\section{Proof of Theorem 1}

The Moore-Penrose inverse of $\bm{B}_i$ can be computed from its eigen-decomposition. Let $b \leq n_i$ denote the rank of $\bm{B}_i$. 
Let $\bs\Lambda$ be the $b \times b$ diagonal matrix of the positive eigenvalues of $\bm{B}_i$ and $\bm{V}$ be the $n_i \times b$ matrix of corresponding eigen-vectors, so that $\bm{B}_i = \bm{V}\bs\Lambda\bm{V}'$. 
Then $\bm{B}_i^+ = \bm{V}\bs\Lambda^{-1}\bm{V}'$ and $\bm{B}_i^{+1/2} = \bm{V}\bs\Lambda^{-1/2}\bm{V}'$. Now, observe that 
\begin{equation}\begin{aligned}
\label{eq:step1}
\bm{\ddot{R}}_i' \bm{W}_i \bm{A}_i \left(\bm{I} - \bm{H_X}\right)_i \bs\Phi \left(\bm{I} - \bm{H_X}\right)_i' \bm{A}_i' \bm{W}_i \bm{\ddot{R}}_i &= \bm{\ddot{R}}_i' \bm{W}_i \bm{D}_i \bm{B}_i^{+1/2} \bm{B}_i \bm{B}_i^{+1/2} \bm{D}_i' \bm{W}_i \bm{\ddot{R}}_i \\
&= \bm{\ddot{R}}_i' \bm{W}_i \bm{D}_i \bm{V}\bm{V}' \bm{D}_i' \bm{W}_i \bm{\ddot{R}}_i. 
\end{aligned}\end{equation}

Because $\bm{D}_i$, and $\bs\Phi$ are positive definite and $\bm{B}_i$ is symmetric, the eigen-vectors $\bm{V}$ define an orthonormal basis for the column span of $\left(\bm{I} - \bm{H_X}\right)_i$.
We now show that $\bm{\ddot{U}}_i$ is in the column space of $\left(\bm{I} - \bm{H_X}\right)_i$. 
Let $\bm{Z}_i$ be an $n_i \times (r + s)$ matrix of zeros. 
Let $\bm{Z}_k = - \bm{\ddot{U}}_k \bm{L}^{-1}\bm{M}_{\bm{\ddot{U}}}^{-1}$, for $k \neq i$ and take $\bm{Z} = \left(\bm{Z}_1',...,\bm{Z}_m'\right)'$. 
Now observe that $\left(\bm{I} - \bm{H_T}\right) \bm{Z} = \bm{Z}$. 
It follows that 
\begin{align*}
\left(\bm{I} - \bm{H_X}\right)_i \bm{Z} &= \left(\bm{I} - \bm{H_{\ddot{U}}}\right)_i \left(\bm{I} - \bm{H_T}\right) \bm{Z} = \left(\bm{I} - \bm{H_{\ddot{U}}}\right)_i \bm{Z} \\
&= \bm{Z}_i - \bm{\ddot{U}}_i\bm{M_{\ddot{U}}}\sum_{k=1}^m \bm{\ddot{U}}_k'\bm{W}_k\bm{Z}_k \\
&= \bm{\ddot{U}}_i\bm{M_{\ddot{U}}} \left(\sum_{k \neq i} \bm{\ddot{U}}_k' \bm{W}_k \bm{\ddot{U}} \right) \bm{L}^{-1}\bm{M}_{\bm{\ddot{U}}}^{-1} = \bm{\ddot{U}}_i.
\end{align*}
Thus, there exists an $N \times (r + s)$ matrix $\bm{Z}$ such that $\left(\bm{I} - \bm{H_{\ddot{X}}}\right)_i \bm{Z} = \bm{\ddot{U}}_i$, i.e., $\bm{\ddot{U}}_i$ is in the column span of $\left(\bm{I} - \bm{H_X}\right)_i$. Because $\bm{D}_i \bm{W}_i$ is positive definite and $\bm{\ddot{R}}_i$ is a sub-matrix of $\bm{\ddot{U}}_i$, $\bm{D}_i\bm{W}_i\bm{\ddot{R}}_i$ is also in the column span of $\left(\bm{I} - \bm{H_X}\right)_i$. It follows that 
\begin{equation}
\label{eq:step2}
\bm{\ddot{R}}_i' \bm{W}_i \bm{D}_i \bm{V}\bm{V}' \bm{D}_i' \bm{W}_i \bm{\ddot{R}}_i = \bm{\ddot{R}}_i' \bm{W}_i \bs\Phi_i \bm{W}_i \bm{\ddot{R}}_i.
\end{equation}
Substituting (\ref{eq:step2}) into (\ref{eq:step1}) demonstrates that $\bm{A}_i$ satisfies the generalized BRL criterion (Eq. 6 of the main paper).

Under the working model, the residuals from cluster $i$ have mean $\bm{0}$ and variance \[
\Var\left(\bm{\ddot{e}}_i\right) = \left(\bm{I} - \bm{H_X}\right)_i \bs\Phi \left(\bm{I} - \bm{H_X}\right)_i',\] 
It follows that 
\begin{align*}
\E\left(\bm{V}^{CR2}\right) &= \bm{M_{\ddot{R}}}\left[\sum_{i=1}^m \bm{\ddot{R}}_i' \bm{W}_i \bm{A}_i \left(\bm{I} - \bm{H_X}\right)_i \bs\Phi \left(\bm{I} - \bm{H_X}\right)_i' \bm{A}_i \bm{W}_i \bm{\ddot{R}}_i \right] \bm{M_{\ddot{R}}} \\
&= \bm{M_{\ddot{R}}}\left[\sum_{i=1}^m \bm{\ddot{R}}_i' \bm{W}_i \bs\Phi_i \bm{W}_i \bm{\ddot{R}}_i \right] \bm{M_{\ddot{R}}} \\
&= \Var\left(\bs{\hat\beta}\right)
\end{align*}

\section{Proof of Theorem 2}

From the fact that $\bm{\ddot{U}}_i'\bm{W}_i\bm{T}_i = \bm{0}$ for $i = 1,...,m$, it follows that \begin{align*}
\bm{B}_i &= \bm{D}_i \left(\bm{I} - \bm{H_{\ddot{U}}}\right)_i \left(\bm{I} - \bm{H_T}\right) \bs\Phi \left(\bm{I} - \bm{H_T}\right)' \left(\bm{I} - \bm{H_{\ddot{U}}}\right)_i' \bm{D}_i'\\
&= \bm{D}_i \left(\bm{I} - \bm{H_{\ddot{U}}} - \bm{H_T}\right)_i \bs\Phi \left(\bm{I} - \bm{H_{\ddot{U}}} - \bm{H_T}\right)_i' \bm{D}_i' \\
&= \bm{D}_i \left(\bs\Phi_i - \bm{\ddot{U}}_i \bm{M_{\ddot{U}}}\bm{\ddot{U}}_i' - \bm{T}_i \bm{M_T}\bm{T}_i'\right)\bm{D}_i'
\end{align*}
and 
\begin{equation}
\label{eq:B_i_inverse}
\bm{B}_i^+ = \left(\bm{D}_i'\right)^{-1} \left(\bs\Phi_i - \bm{\ddot{U}}_i \bm{M_{\ddot{U}}}\bm{\ddot{U}}_i' - \bm{T}_i \bm{M_T}\bm{T}_i'\right)^+ \bm{D}_i^{-1}.
\end{equation}
Let $\bs\Psi_i = \left(\bs\Phi_i - \bm{\ddot{U}}_i \bm{M_{\ddot{U}}}\bm{\ddot{U}}_i'\right)^+$.
Using a generalized Woodbury identity \citep{Henderson1981on}, \[
\bs\Psi_i = \bm{W}_i + \bm{W}_i \bm{\ddot{U}}_i \bm{M_{\ddot{U}}}\left(\bm{M_{\ddot{U}}} - \bm{M_{\ddot{U}}} \bm{\ddot{U}}_i' \bm{W}_i \bm{\ddot{U}}_i \bm{M_{\ddot{U}}}\right)^+ \bm{M_{\ddot{U}}}\bm{\ddot{U}}_i'\bm{W}_i. \]
It follows that $\bs\Psi_i \bm{T}_i = \bm{W}_i \bm{T}_i$. 
Another application of the generalized Woodbury identity gives 
\begin{align*}
\left(\bs\Phi_i - \bm{\ddot{U}}_i \bm{M_{\ddot{U}}}\bm{\ddot{U}}_i' - \bm{T}_i \bm{M_T}\bm{T}_i'\right)^+ &= \bs\Psi_i + \bs\Psi_i \bm{T}_i \bm{M_T}\left(\bm{M_T} - \bm{M_T}\bm{T}_i' \bs\Psi_i \bm{T}_i \bm{M_T}\right)^+ \bm{M_T} \bm{T}_i' \bs\Psi_i \\
&= \bs\Psi_i + \bm{W}_i \bm{T}_i \bm{M_T}\left(\bm{M_T} - \bm{M_T}\bm{T}_i' \bm{W}_i \bm{T}_i\bm{M_T}\right)^+ \bm{M_T} \bm{T}_i' \bm{W}_i \\
&= \bs\Psi_i.
\end{align*}
The last equality follows from the fact that \[
\bm{T}_i \bm{M_T}\left(\bm{M_T} - \bm{M_T}\bm{T}_i' \bm{W}_i \bm{T}_i\bm{M_T}\right)^{-} \bm{M_T} \bm{T}_i' = \bm{0} \]
because the fixed effects are nested within clusters. 
Substituting into (\ref{eq:B_i_inverse}), we then have that $\bm{B}_i^+ = \left(\bm{D}_i'\right)^{-1} \bs\Psi_i \bm{D}_i^{-1}$. 
But \[
\bm{\tilde{B}}_i = \bm{D}_i \left(\bm{I} - \bm{H_{\ddot{U}}}\right)_i \bs\Phi \left(\bm{I} - \bm{H_{\ddot{U}}}\right)_i' \bm{D}_i' = \bm{D}_i \left(\bs\Phi_i - \bm{\ddot{U}}_i\bm{M_{\ddot{U}}} \bm{\ddot{U}}_i'\right) \bm{D}_i' = \bm{D}_i \bs\Psi_i^+ \bm{D}_i',
\]
and so $\bm{B}_i^+ = \bm{\tilde{B}}_i^+$. It follows that $\bm{A}_i = \bm{\tilde{A}}_i$ for $i = 1,...,m$. 

\newpage
\section{Details of simulation study}
\label{sec:simulations}

This section provides further details regarding the design of the simulations reported in Section 4 of the main text. The simulations examined six distinct study designs. Outcomes are measured for $n$ units (which may be individuals, as in a cluster-randomized or block-randomized design, or time-points, as in a difference-in-differences panel) in each of $m$ clusters under one of three treatment conditions. Suppose that there are $G$ sets of clusters, each of size $m_g$, where the clusters in each set have a distinct pattern of treatment assignments. Let $n_{ghi}$ denote the number of units at which cluster $i$ in group $g$ is observed under condition $h$, for $i=1,...,m$, $g = 1,...,G$, and $h = 1,2,3$. The following six designs were simulated:
\begin{enumerate}
\item A balanced, block-randomized design, with an un-equal allocation within each block. In the balanced design, the treatment allocation is identical for each block, with $G = 1$, $m_1 = m$, $n_{11i} = n / 2$, $n_{12i} = n / 3$, and $n_{13i} = n / 6$.
\item An unbalanced, block-randomized design, with two different patterns of treatment allocation. Here, $G = 2$,  $m_1 = m_2 = m / 2$, $n_{11i} = n / 2$, $n_{12i} = n / 3$, $n_{13i} = n / 6$, $n_{21i} = n / 3$, $n_{22i} = 5 n / 9$, and $n_{23i} = n / 9$.
\item A balanced, cluster-randomized design, in which units are nested within clusters and an equal number of clusters are assigned to each treatment condition. Here, $G = 3$, $m_g = m / 3$, and $n_{ghi} = n$ for $g = h$ and zero otherwise.
\item An unbalanced, cluster-randomized design, in which units are nested within clusters but the number of clusters assigned to each condition is not equal. Here, $G = 3$; $m_1 = 0.5 m, m_2 = 0.3 m, m_3 = 0.2 m$; and $n_{ghi} = n$ for $g = h$ and zero otherwise. 
\item A balanced difference-in-differences design, with two patterns of treatment allocation ($G = 2$) and clusters allocated equally to each pattern ($m_1 = m_2 = m / 2$). Here, half of the clusters are observed under the first treatment condition only ($n_{11i} = n$) and the remaining half are observed under all three conditions, with $n_{21i} = n / 2$, $n_{22i} = n / 3$, and $n_{23i} = n / 6$.
\item An unbalanced difference-in-differences design, again with two patterns of treatment allocation ($G = 2$), but where $m_1 = 2 m / 3$ clusters are observed under the first treatment condition only ($n_{11i} = n$) and the remaining $m_2 = m / 3$ clusters are observed under all three conditions, with  $n_{21i} = n / 2$, $n_{22i} = n / 3$, and $n_{23i} = n / 6$.
\end{enumerate}

\begin{landscape}
\section{Additional simulation results}



\begin{knitrout}
\definecolor{shadecolor}{rgb}{0.969, 0.969, 0.969}\color{fgcolor}\begin{figure}[H]

{\centering \includegraphics[width=\linewidth]{CR_fig/overview_01-1} 

}

\caption[Rejection rates of AHT and standard tests for ]{Rejection rates of AHT and standard tests for $\alpha = .01$, by dimension of hypothesis ($q$) and sample size ($m$).}\label{fig:overview_01}
\end{figure}


\end{knitrout}

\begin{knitrout}
\definecolor{shadecolor}{rgb}{0.969, 0.969, 0.969}\color{fgcolor}\begin{figure}[H]

{\centering \includegraphics[width=\linewidth]{CR_fig/overview_10-1} 

}

\caption[Rejection rates of AHT and standard tests for ]{Rejection rates of AHT and standard tests for $\alpha = .10$, by dimension of hypothesis ($q$) and sample size ($m$).}\label{fig:overview_10}
\end{figure}


\end{knitrout}

\begin{knitrout}
\definecolor{shadecolor}{rgb}{0.969, 0.969, 0.969}\color{fgcolor}\begin{figure}[H]

{\centering \includegraphics[width=\linewidth]{CR_fig/balance_01_15-1} 

}

\caption[Rejection rates of AHT and standard tests, by study design and dimension of hypothesis (]{Rejection rates of AHT and standard tests, by study design and dimension of hypothesis ($q$) for $\alpha = .01$ and $m = 15$. CR = cluster-randomized design; DD = difference-in-differences design; RB = randomized block design; B = balanced; U = unbalanced.}\label{fig:balance_01_15}
\end{figure}


\end{knitrout}

\begin{knitrout}
\definecolor{shadecolor}{rgb}{0.969, 0.969, 0.969}\color{fgcolor}\begin{figure}[H]

{\centering \includegraphics[width=\linewidth]{CR_fig/balance_05_15-1} 

}

\caption[Rejection rates of AHT and standard tests, by study design and dimension of hypothesis (]{Rejection rates of AHT and standard tests, by study design and dimension of hypothesis ($q$) for $\alpha = .05$ and $m = 15$. CR = cluster-randomized design; DD = difference-in-differences design; RB = randomized block design; B = balanced; U = unbalanced.}\label{fig:balance_05_15}
\end{figure}


\end{knitrout}

\begin{knitrout}
\definecolor{shadecolor}{rgb}{0.969, 0.969, 0.969}\color{fgcolor}\begin{figure}[H]

{\centering \includegraphics[width=\linewidth]{CR_fig/balance_10_15-1} 

}

\caption[Rejection rates of AHT and standard tests, by study design and dimension of hypothesis (]{Rejection rates of AHT and standard tests, by study design and dimension of hypothesis ($q$) for $\alpha = .10$ and $m = 15$. CR = cluster-randomized design; DD = difference-in-differences design; RB = randomized block design; B = balanced; U = unbalanced.}\label{fig:balance_10_15}
\end{figure}


\end{knitrout}

\begin{knitrout}
\definecolor{shadecolor}{rgb}{0.969, 0.969, 0.969}\color{fgcolor}\begin{figure}[H]

{\centering \includegraphics[width=\linewidth]{CR_fig/balance_01_30-1} 

}

\caption[Rejection rates of AHT and standard tests, by study design and dimension of hypothesis (]{Rejection rates of AHT and standard tests, by study design and dimension of hypothesis ($q$) for $\alpha = .01$ and $m = 30$. CR = cluster-randomized design; DD = difference-in-differences design; RB = randomized block design; B = balanced; U = unbalanced.}\label{fig:balance_01_30}
\end{figure}


\end{knitrout}

\begin{knitrout}
\definecolor{shadecolor}{rgb}{0.969, 0.969, 0.969}\color{fgcolor}\begin{figure}[H]

{\centering \includegraphics[width=\linewidth]{CR_fig/balance_10_30-1} 

}

\caption[Rejection rates of AHT and standard tests, by study design and dimension of hypothesis (]{Rejection rates of AHT and standard tests, by study design and dimension of hypothesis ($q$) for $\alpha = .10$ and $m = 30$. CR = cluster-randomized design; DD = difference-in-differences design; RB = randomized block design; B = balanced; U = unbalanced.}\label{fig:balance_10_30}
\end{figure}


\end{knitrout}

\begin{knitrout}
\definecolor{shadecolor}{rgb}{0.969, 0.969, 0.969}\color{fgcolor}\begin{figure}[H]

{\centering \includegraphics[width=\linewidth]{CR_fig/balance_01_50-1} 

}

\caption[Rejection rates of AHT and standard tests, by study design and dimension of hypothesis (]{Rejection rates of AHT and standard tests, by study design and dimension of hypothesis ($q$) for $\alpha = .01$ and $m = 50$. CR = cluster-randomized design; DD = difference-in-differences design; RB = randomized block design; B = balanced; U = unbalanced.}\label{fig:balance_01_50}
\end{figure}


\end{knitrout}

\begin{knitrout}
\definecolor{shadecolor}{rgb}{0.969, 0.969, 0.969}\color{fgcolor}\begin{figure}[H]

{\centering \includegraphics[width=\linewidth]{CR_fig/balance_05_50-1} 

}

\caption[Rejection rates of AHT and standard tests, by study design and dimension of hypothesis (]{Rejection rates of AHT and standard tests, by study design and dimension of hypothesis ($q$) for $\alpha = .05$ and $m = 50$. CR = cluster-randomized design; DD = difference-in-differences design; RB = randomized block design; B = balanced; U = unbalanced.}\label{fig:balance_05_50}
\end{figure}


\end{knitrout}

\begin{knitrout}
\definecolor{shadecolor}{rgb}{0.969, 0.969, 0.969}\color{fgcolor}\begin{figure}[H]

{\centering \includegraphics[width=\linewidth]{CR_fig/balance_10_50-1} 

}

\caption[Rejection rates of AHT and standard tests, by study design and dimension of hypothesis (]{Rejection rates of AHT and standard tests, by study design and dimension of hypothesis ($q$) for $\alpha = .10$ and $m = 50$. CR = cluster-randomized design; DD = difference-in-differences design; RB = randomized block design; B = balanced; U = unbalanced.}\label{fig:balance_10_50}
\end{figure}


\end{knitrout}

\begin{knitrout}
\definecolor{shadecolor}{rgb}{0.969, 0.969, 0.969}\color{fgcolor}\begin{figure}[H]

{\centering \includegraphics[width=\linewidth]{CR_fig/misspecification_01-1} 

}

\caption[Rejection rates of AHT test, by treatment effect variance and intra-class correlation for ]{Rejection rates of AHT test, by treatment effect variance and intra-class correlation for $\alpha = .01$.}\label{fig:misspecification_01}
\end{figure}


\end{knitrout}

\begin{knitrout}
\definecolor{shadecolor}{rgb}{0.969, 0.969, 0.969}\color{fgcolor}\begin{figure}[H]

{\centering \includegraphics[width=\linewidth]{CR_fig/misspecification_05-1} 

}

\caption[Rejection rates of AHT test, by treatment effect variance and intra-class correlation for ]{Rejection rates of AHT test, by treatment effect variance and intra-class correlation for $\alpha = .05$.}\label{fig:misspecification_05}
\end{figure}


\end{knitrout}

\begin{knitrout}
\definecolor{shadecolor}{rgb}{0.969, 0.969, 0.969}\color{fgcolor}\begin{figure}[H]

{\centering \includegraphics[width=\linewidth]{CR_fig/misspecification_10-1} 

}

\caption[Rejection rates of AHT test, by treatment effect variance and intra-class correlation for ]{Rejection rates of AHT test, by treatment effect variance and intra-class correlation for $\alpha = .10$.}\label{fig:misspecification_10}
\end{figure}


\end{knitrout}
\end{landscape}

\bibliographystyle{agsm}
\bibliography{bibliography}

\end{document}
