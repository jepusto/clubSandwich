% !TeX program = pdfLaTeX
\documentclass[12pt]{article}
\usepackage{amsmath}
\usepackage{graphicx,psfrag,epsf}
\usepackage{enumerate}


\usepackage{textcomp}
\usepackage[hyphens]{url} % not crucial - just used below for the URL
\usepackage{hyperref}

%\pdfminorversion=4
% NOTE: To produce blinded version, replace "0" with "1" below.
\newcommand{\blind}{0}

% DON'T change margins - should be 1 inch all around.
\addtolength{\oddsidemargin}{-.5in}%
\addtolength{\evensidemargin}{-.5in}%
\addtolength{\textwidth}{1in}%
\addtolength{\textheight}{1.3in}%
\addtolength{\topmargin}{-.8in}%

%% load any required packages here


% Pandoc syntax highlighting
\usepackage{color}
\usepackage{fancyvrb}
\newcommand{\VerbBar}{|}
\newcommand{\VERB}{\Verb[commandchars=\\\{\}]}
\DefineVerbatimEnvironment{Highlighting}{Verbatim}{commandchars=\\\{\}}
% Add ',fontsize=\small' for more characters per line
\usepackage{framed}
\definecolor{shadecolor}{RGB}{248,248,248}
\newenvironment{Shaded}{\begin{snugshade}}{\end{snugshade}}
\newcommand{\AlertTok}[1]{\textcolor[rgb]{0.94,0.16,0.16}{#1}}
\newcommand{\AnnotationTok}[1]{\textcolor[rgb]{0.56,0.35,0.01}{\textbf{\textit{#1}}}}
\newcommand{\AttributeTok}[1]{\textcolor[rgb]{0.77,0.63,0.00}{#1}}
\newcommand{\BaseNTok}[1]{\textcolor[rgb]{0.00,0.00,0.81}{#1}}
\newcommand{\BuiltInTok}[1]{#1}
\newcommand{\CharTok}[1]{\textcolor[rgb]{0.31,0.60,0.02}{#1}}
\newcommand{\CommentTok}[1]{\textcolor[rgb]{0.56,0.35,0.01}{\textit{#1}}}
\newcommand{\CommentVarTok}[1]{\textcolor[rgb]{0.56,0.35,0.01}{\textbf{\textit{#1}}}}
\newcommand{\ConstantTok}[1]{\textcolor[rgb]{0.00,0.00,0.00}{#1}}
\newcommand{\ControlFlowTok}[1]{\textcolor[rgb]{0.13,0.29,0.53}{\textbf{#1}}}
\newcommand{\DataTypeTok}[1]{\textcolor[rgb]{0.13,0.29,0.53}{#1}}
\newcommand{\DecValTok}[1]{\textcolor[rgb]{0.00,0.00,0.81}{#1}}
\newcommand{\DocumentationTok}[1]{\textcolor[rgb]{0.56,0.35,0.01}{\textbf{\textit{#1}}}}
\newcommand{\ErrorTok}[1]{\textcolor[rgb]{0.64,0.00,0.00}{\textbf{#1}}}
\newcommand{\ExtensionTok}[1]{#1}
\newcommand{\FloatTok}[1]{\textcolor[rgb]{0.00,0.00,0.81}{#1}}
\newcommand{\FunctionTok}[1]{\textcolor[rgb]{0.00,0.00,0.00}{#1}}
\newcommand{\ImportTok}[1]{#1}
\newcommand{\InformationTok}[1]{\textcolor[rgb]{0.56,0.35,0.01}{\textbf{\textit{#1}}}}
\newcommand{\KeywordTok}[1]{\textcolor[rgb]{0.13,0.29,0.53}{\textbf{#1}}}
\newcommand{\NormalTok}[1]{#1}
\newcommand{\OperatorTok}[1]{\textcolor[rgb]{0.81,0.36,0.00}{\textbf{#1}}}
\newcommand{\OtherTok}[1]{\textcolor[rgb]{0.56,0.35,0.01}{#1}}
\newcommand{\PreprocessorTok}[1]{\textcolor[rgb]{0.56,0.35,0.01}{\textit{#1}}}
\newcommand{\RegionMarkerTok}[1]{#1}
\newcommand{\SpecialCharTok}[1]{\textcolor[rgb]{0.00,0.00,0.00}{#1}}
\newcommand{\SpecialStringTok}[1]{\textcolor[rgb]{0.31,0.60,0.02}{#1}}
\newcommand{\StringTok}[1]{\textcolor[rgb]{0.31,0.60,0.02}{#1}}
\newcommand{\VariableTok}[1]{\textcolor[rgb]{0.00,0.00,0.00}{#1}}
\newcommand{\VerbatimStringTok}[1]{\textcolor[rgb]{0.31,0.60,0.02}{#1}}
\newcommand{\WarningTok}[1]{\textcolor[rgb]{0.56,0.35,0.01}{\textbf{\textit{#1}}}}

% tightlist command for lists without linebreak
\providecommand{\tightlist}{%
  \setlength{\itemsep}{0pt}\setlength{\parskip}{0pt}}


% Pandoc citation processing
\newlength{\cslhangindent}
\setlength{\cslhangindent}{1.5em}
\newlength{\csllabelwidth}
\setlength{\csllabelwidth}{3em}
\newlength{\cslentryspacingunit} % times entry-spacing
\setlength{\cslentryspacingunit}{\parskip}
% for Pandoc 2.8 to 2.10.1
\newenvironment{cslreferences}%
  {}%
  {\par}
% For Pandoc 2.11+
\newenvironment{CSLReferences}[2] % #1 hanging-ident, #2 entry spacing
 {% don't indent paragraphs
  \setlength{\parindent}{0pt}
  % turn on hanging indent if param 1 is 1
  \ifodd #1
  \let\oldpar\par
  \def\par{\hangindent=\cslhangindent\oldpar}
  \fi
  % set entry spacing
  \setlength{\parskip}{#2\cslentryspacingunit}
 }%
 {}
\usepackage{calc}
\newcommand{\CSLBlock}[1]{#1\hfill\break}
\newcommand{\CSLLeftMargin}[1]{\parbox[t]{\csllabelwidth}{#1}}
\newcommand{\CSLRightInline}[1]{\parbox[t]{\linewidth - \csllabelwidth}{#1}\break}
\newcommand{\CSLIndent}[1]{\hspace{\cslhangindent}#1}

\usepackage{amsthm}
\newtheorem*{thm}{Theorem}
\newtheorem{lem}{Lemma}
\usepackage{booktabs}
\usepackage{longtable}
\usepackage[natbibapa]{apacite}
\usepackage{caption}
\usepackage{multirow}
\usepackage{float}    % for fig.pos='H'
\usepackage{rotfloat} % for sidewaysfigure
\newcommand{\Prob}{\text{Pr}}
\newcommand{\E}{\text{E}}
\newcommand{\Cov}{\text{Cov}}
\newcommand{\corr}{\text{corr}}
\newcommand{\Var}{\text{Var}}
\newcommand{\iid}{\stackrel{\text{iid}}{\sim}}
\newcommand{\tr}{\text{tr}}
\newcommand{\bm}{\mathbf}
\newcommand{\bs}{\boldsymbol}

\begin{document}


\def\spacingset#1{\renewcommand{\baselinestretch}%
{#1}\small\normalsize} \spacingset{1}


%%%%%%%%%%%%%%%%%%%%%%%%%%%%%%%%%%%%%%%%%%%%%%%%%%%%%%%%%%%%%%%%%%%%%%%%%%%%%%

\if0\blind
{
  \title{\bf Corrigendum: Small sample methods for cluster-robust
variance estimation and hypothesis testing in fixed effects models}

  \author{
        James E. Pustejovsky \thanks{Department of Educational
Psychology, University of Wisconsin - Madison, 1025 West Johnson Street,
Madison, WI 53706. Email:
\href{mailto:pustejovsky@wisc.edu}{\nolinkurl{pustejovsky@wisc.edu}}} \\
    University of Wisconsin - Madison\\
     and \\     Elizabeth Tipton \thanks{Department of Statistics,
Northwestern University. Email:
\href{mailto:tipton@northwestern.edu}{\nolinkurl{tipton@northwestern.edu}}} \\
    Northwestern University\\
      }
  \maketitle
} \fi

\if1\blind
{
  \bigskip
  \bigskip
  \bigskip
  \begin{center}
    {\LARGE\bf Corrigendum: Small sample methods for cluster-robust
variance estimation and hypothesis testing in fixed effects models}
  \end{center}
  \medskip
} \fi

\bigskip
\begin{abstract}
Pustejovsky and Tipton (2018) considered how to implement cluster-robust
variance estimators for fixed effects models estimated by weighted (or
unweighted) least squares. Theorem 2 of the paper concerns a
computational short cut for a certain cluster-robust variance estimator
in models with cluster-specific fixed effects. It claimed that this
short cut works for models estimated by generalized least squares, as
long as the weights are taken to be inverse of the working model.
However, the theorem is incorrect. In this corrigendum, we review the
CR2 variance estimator, describe the assertion of the theorem as
originally stated, and explain the error. We then provide a revised
version of the theorem, which holds only under more limited conditions,
for models estimated by ordinary least squares.
\end{abstract}

\noindent%
{\it Keywords:} 
\vfill

\newpage
\spacingset{1.45} % DON'T change the spacing!

\hypertarget{a-fixed-effects-model}{%
\section{A fixed effects model}\label{a-fixed-effects-model}}

For data that can be grouped into \(m\) clusters of observations,
Pustejovsky and Tipton (2018) considered the model \begin{equation}
\label{eq:regression}
\mathbf{y}_i = \mathbf{R}_i \boldsymbol\beta + \mathbf{S}_i \boldsymbol\gamma + \mathbf{T}_i \boldsymbol\mu + \boldsymbol\epsilon_i,
\end{equation} where \(\mathbf{y}_i\) is an \(n_i \times 1\) vector of
responses for cluster \(i\), \(\mathbf{R}_i\) is an \(n_i \times r\)
matrix of focal predictors, \(\mathbf{S}_i\) is an \(n_i \times s\)
matrix of additional covariates that vary across multiple clusters, and
\(\mathbf{T}_i\) is an \(n_i \times t\) matrix encoding cluster-specific
fixed effects, all for \(i = 1,...,m\). The cluster-specific fixed
effects satisfy \(\mathbf{T}_h \mathbf{T}_i' = \mathbf{0}\) for
\(h \neq i\). Interest centers on inference for the coefficients on the
focal predictors \(\boldsymbol\beta\).

Pustejovsky and Tipton (2018) considered estimation of Model
\ref{eq:regression} by weighted least squares (WLS). Let
\(\mathbf{W}_1,...,\mathbf{W}_m\) be a set of symmetric weight matrices
used for WLS estimation. The CR2 variance estimator involves specifying
a working model for the structure of the errors. Consider a working
model
\(\text{Var}\left(\boldsymbol\epsilon_i | \mathbf{R}_i, \mathbf{S}_i, \mathbf{T}_i\right) = \sigma^2 \boldsymbol\Phi_i\)
where \(\boldsymbol\Phi_i\) is a symmetric \(n_i \times n_i\) matrix
that may be a function of a low-dimensional, estimable parameter. In
some applications, the weight matrices might be taken as
\(\mathbf{W}_i = \boldsymbol{\hat\Phi}_i^{-1}\), where
\(\boldsymbol{\hat\Phi}_i\) is an estimate of \(\boldsymbol\Phi_i\). In
other applications, the weight matrices may be something else, such as
diagonal matrices consisting of sampling weights or identity matrices
(i.e., ordinary least squares).

\hypertarget{the-cr2-variance-estimator}{%
\section{The CR2 variance estimator}\label{the-cr2-variance-estimator}}

Pustejovsky and Tipton (2018) provided a generalization of the
bias-reduced linearization estimator introduced by McCaffrey, Bell, and
Botts (2001) and Bell and McCaffrey (2002) that can be applied to Model
\ref{eq:regression}, referred to as the CR2 variance estimator. We
follow the same notation as Pustejovsky and Tipton (2018) to define CR2.
Let \(N = \sum_{i=1}^m n_i\) be the total sample size. Let
\(\mathbf{U}_i = \left[ \mathbf{R}_i \ \mathbf{S}_i \right]\) be the set
of predictors that vary across clusters and
\(\mathbf{X}_i = \left[ \mathbf{R}_i \ \mathbf{S}_i \ \mathbf{T}_i \right]\)
be the full set of predictors. Let \(\mathbf{R}\), \(\mathbf{S}\),
\(\mathbf{T}\), \(\mathbf{U}\), \(\mathbf{X}\), and \(\mathbf{y}\)
denote the stacked versions of the cluster-specific matrices (i.e.,
\(\mathbf{R} = \left[\mathbf{R}_1' \ \mathbf{R}_2' \ \cdots \ \mathbf{R}_m'\right]'\),
etc.). Let \(\mathbf{W} = \bigoplus_{i=1}^m \mathbf{W}_i\) and
\(\boldsymbol\Phi = \bigoplus_{i=1}^m \boldsymbol\Phi_i\). For a generic
matrix \(\mathbf{Z}\), let
\(\mathbf{M}_{Z} = \left(\mathbf{Z}'\mathbf{W}\mathbf{Z}\right)^{-1}\)
and
\(\mathbf{H}_{\mathbf{Z}} = \mathbf{Z} \mathbf{M}_{\mathbf{Z}}\mathbf{Z}'\mathbf{W}\).
Let \(\mathbf{C}_i\) be the \(n_i \times N\) matrix that selects the
rows of cluster \(i\) from the full set of observations, such that
\(\mathbf{X}_i = \mathbf{C}_i \mathbf{X}\). Finally, let
\(\mathbf{D}_i\) be the upper-right Cholesky factorization of
\(\mathbf{\Phi}_i\).

These operators provide a means to define absorbed versions of the
predictors and the outcome. Let
\(\mathbf{\ddot{S}} = \left(\mathbf{I} - \mathbf{H}_{\mathbf{T}}\right) \mathbf{S}\)
be the covariates after absorbing the cluster-specific effects, let
\(\mathbf{\ddot{U}} = \left(\mathbf{I} - \mathbf{H}_{\mathbf{T}}\right) \mathbf{U}\)
be an absorbed version of the focal predictors and the covariates, let
\(\mathbf{\ddot{R}} = \left(\mathbf{I} - \mathbf{H}_{\mathbf{\ddot{S}}}\right)\left(\mathbf{I} - \mathbf{H}_{\mathbf{T}}\right) \mathbf{R}\)
be the focal predictors after absorbing the covariates and the
cluster-specific fixed effects, and let
\(\mathbf{e} = \left(\mathbf{I} - \mathbf{H}_{\mathbf{\ddot{R}}}\right)\left(\mathbf{I} - \mathbf{H}_{\mathbf{\ddot{S}}}\right)\left(\mathbf{I} - \mathbf{H}_{\mathbf{T}}\right) \mathbf{y}\)
denote the vector of residuals, with
\(\mathbf{e}_i = \mathbf{C}_i\mathbf{e}\).

With this notation established, the CR2 variance estimator has the form
\begin{equation}
\mathbf{V}^{CR2} = \mathbf{M}_{\mathbf{\ddot{R}}} \left(\sum_{i=1}^m \mathbf{\ddot{R}}_i' \mathbf{W}_i \mathbf{A}_i \mathbf{e}_i \mathbf{e}_i' \mathbf{A}_i \mathbf{W}_i \mathbf{\ddot{R}}_i \right) \mathbf{M}_{\mathbf{\ddot{R}}},
\end{equation} where
\(\mathbf{\ddot{R}}_i = \mathbf{C}_i \mathbf{\ddot{R}}\) is the
cluster-specific matrix of absorbed focal predictors, \(\mathbf{e}_i\)
is the vector of weighted least squares residuals from cluster \(i\),
and \(\mathbf{A}_1,...,\mathbf{A}_m\) are a set of adjustment matrices
that correct the bias of the residual cross-products.

The adjustment matrices are calculated as follows. Define the matrices
\begin{equation}
\label{eq:B-matrix}
\mathbf{B}_i = \mathbf{D}_i \mathbf{C}_i \left(\mathbf{I} - \mathbf{H}_{\mathbf{X}}\right) \boldsymbol\Phi \left(\mathbf{I} - \mathbf{H}_{\mathbf{X}}\right)'\mathbf{C}_i' \mathbf{D}_i'
\end{equation} for \(i = 1,...,m\). The adjustment matrices are then
calculated as \begin{equation}
\label{eq:A-matrix}
\mathbf{A}_i = \mathbf{D}_i' \mathbf{B}_i^{+1/2} \mathbf{D}_i,
\end{equation} where \(\mathbf{B}_i^{+1/2}\) is the symmetric square
root of the Moore-Penrose inverse of \(\mathbf{B}_i\). Theorem 1 of
Pustejovsky and Tipton (2018) shows that, if the working model
\(\boldsymbol\Phi\) is correctly specified and some conditions on the
rank of \(\mathbf{U}\) are satisfied, then the CR2 estimator is exactly
unbiased for the sampling variance of the weighted least squares
estimator of \(\boldsymbol\beta\). Moreover, although the CR2 estimator
is defined based on a working model, it remains close to unbiased and
outperforms alternative sandwich estimators even when the working model
is not correctly specified.

\hypertarget{the-original-statement-of-theorem-2}{%
\section{The original statement of Theorem
2}\label{the-original-statement-of-theorem-2}}

The adjustment matrices given in (\ref{eq:A-matrix}) can be expensive to
compute directly because the \(\mathbf{B}_i\) matrices involve computing
a ``residualized'' version of the \(N \times N\) matrix
\(\boldsymbol\Phi\) involving the full set of predictors
\(\mathbf{X}\)---including the cluster-specific fixed effects
\(\mathbf{T}_1,...,\mathbf{T}_m\). Theorem 2 considered whether one can
take a computational short cut by omitting the cluster-specific fixed
effects from the calculation of the \(\mathbf{B}_i\) matrices.
Specifically, define the modified matrices \begin{equation}
\label{eq:B-modified}
\mathbf{\tilde{B}}_i = \mathbf{D}_i \mathbf{C}_i \left(\mathbf{I} - \mathbf{H}_{\mathbf{\ddot{U}}}\right) \boldsymbol\Phi \left(\mathbf{I} - \mathbf{H}_{\mathbf{\ddot{U}}}\right)'\mathbf{C}_i' \mathbf{D}_i'
\end{equation} and \begin{equation}
\label{eq:A-modified}
\mathbf{\tilde{A}}_i = \mathbf{D}_i' \mathbf{\tilde{B}}_i^{+1/2} \mathbf{D}_i.
\end{equation} Theorem 2 claimed that if the weight matrices are inverse
of the working model, such that
\(\mathbf{W}_i = \boldsymbol\Phi_i^{-1}\) for \(i = 1,...,m\), then
\(\mathbf{\tilde{B}}_i^{+1/2} = \mathbf{B}_i^{+1/2}\) and hence
\(\mathbf{\tilde{A}}_i = \mathbf{A}_i\). The implication is that the
cluster-specific fixed effects can be ignored when calculating the
adjustment matrices. However, the claimed equivalence does not actually
hold. The proof of Theorem 2 as given in the supplementary materials of
Pustejovsky and Tipton (2018) relied on a Woodbury identity for
generalized inverses that does not hold for \(\mathbf{B}_i\) because
necessary rank conditions are not satisfied.

We describe a simple numerical example that contradicts the original
statement of Theorem 2. Consider a block-randomized experiment with
\(m = 3\) blocks, of sizes \(n_1 = 2\), \(n_2 = 4\), and \(n_3 = 6\). In
each block, a single observation receives treatment and the remaining
observations receive the control condition, so that
\(\mathbf{R}_i = \left[1 \ 0 \cdots 0\right]'\), \(\mathbf{T}\) consists
of indicators for each block, and there are no additional covariates.
The model is then \[
\mathbf{y}_i = \mathbf{R}_i \beta + \mu_i + \boldsymbol\epsilon_i
\] Assume that treatment impacts are heterogeneous such that
\(\text{Var}(\boldsymbol\epsilon_{1i}) = \sigma^2 + \tau^2\) and
\(\text{Var}(\boldsymbol\epsilon_{ji}) = \sigma^2\) for \(j > 1\), where
\(\tau^2\) is known. Consider estimating the model using weighted least
squares with inverse-variance weights.

\begin{Shaded}
\begin{Highlighting}[]
\FunctionTok{library}\NormalTok{(clubSandwich)}
\end{Highlighting}
\end{Shaded}

\begin{verbatim}
## Registered S3 method overwritten by 'clubSandwich':
##   method    from    
##   bread.mlm sandwich
\end{verbatim}

\begin{Shaded}
\begin{Highlighting}[]
\NormalTok{ni }\OtherTok{\textless{}{-}} \FunctionTok{c}\NormalTok{(}\DecValTok{2}\NormalTok{,}\DecValTok{4}\NormalTok{,}\DecValTok{6}\NormalTok{)}
\NormalTok{m }\OtherTok{\textless{}{-}} \FunctionTok{length}\NormalTok{(ni)}
\NormalTok{ri }\OtherTok{\textless{}{-}} \FunctionTok{lapply}\NormalTok{(ni, \textbackslash{}(x) }\FunctionTok{c}\NormalTok{(1L, }\FunctionTok{rep}\NormalTok{(0L, x }\SpecialCharTok{{-}}\NormalTok{ 1L)))}
\NormalTok{wi }\OtherTok{\textless{}{-}} \FunctionTok{lapply}\NormalTok{(ri, \textbackslash{}(r) }\FunctionTok{length}\NormalTok{(r) }\SpecialCharTok{*}\NormalTok{ (r }\SpecialCharTok{+}\NormalTok{ (}\DecValTok{1} \SpecialCharTok{{-}}\NormalTok{ r) }\SpecialCharTok{/}\NormalTok{ (}\FunctionTok{length}\NormalTok{(r) }\SpecialCharTok{{-}} \DecValTok{1}\NormalTok{)))}
\NormalTok{rddi\_ols }\OtherTok{\textless{}{-}} \FunctionTok{lapply}\NormalTok{(ri, \textbackslash{}(r) r }\SpecialCharTok{{-}} \FunctionTok{mean}\NormalTok{(r))}
\NormalTok{rddi\_wt }\OtherTok{\textless{}{-}} \FunctionTok{lapply}\NormalTok{(ri, \textbackslash{}(r) r }\SpecialCharTok{{-}} \DecValTok{1} \SpecialCharTok{/} \DecValTok{2}\NormalTok{)}
\NormalTok{clust }\OtherTok{\textless{}{-}} \FunctionTok{rep}\NormalTok{(LETTERS[}\DecValTok{1}\SpecialCharTok{:}\NormalTok{m], ni)}
\NormalTok{yi }\OtherTok{\textless{}{-}} \FunctionTok{rnorm}\NormalTok{(}\FunctionTok{sum}\NormalTok{(ni))}
\NormalTok{yddi }\OtherTok{\textless{}{-}} \FunctionTok{tapply}\NormalTok{(yi, clust, \textbackslash{}(x) x }\SpecialCharTok{{-}} \FunctionTok{mean}\NormalTok{(x))}
\NormalTok{dat }\OtherTok{\textless{}{-}} \FunctionTok{data.frame}\NormalTok{(}
  \AttributeTok{y =}\NormalTok{ yi, }
  \AttributeTok{y\_dd =} \FunctionTok{unlist}\NormalTok{(yddi),}
  \AttributeTok{R =} \FunctionTok{unlist}\NormalTok{(ri),}
  \AttributeTok{R\_dd\_ols =} \FunctionTok{unlist}\NormalTok{(rddi\_ols),}
  \AttributeTok{R\_dd\_wt =} \FunctionTok{unlist}\NormalTok{(rddi\_wt),}
  \AttributeTok{w =} \FunctionTok{unlist}\NormalTok{(wi),}
  \AttributeTok{clust =}\NormalTok{ clust}
\NormalTok{)}

\NormalTok{ols\_fit }\OtherTok{\textless{}{-}} \FunctionTok{lm}\NormalTok{(y }\SpecialCharTok{\textasciitilde{}}\NormalTok{ clust }\SpecialCharTok{+}\NormalTok{ R, }\AttributeTok{data =}\NormalTok{ dat)}
\NormalTok{A\_ols\_full }\OtherTok{\textless{}{-}} \FunctionTok{attr}\NormalTok{(}\FunctionTok{vcovCR}\NormalTok{(ols\_fit, }\AttributeTok{type =} \StringTok{"CR2"}\NormalTok{, }\AttributeTok{cluster =}\NormalTok{ dat}\SpecialCharTok{$}\NormalTok{clust),}\StringTok{"adjustments"}\NormalTok{)}
\NormalTok{emat\_ols\_full }\OtherTok{\textless{}{-}} \FunctionTok{mapply}\NormalTok{(\textbackslash{}(r, a) a }\SpecialCharTok{\%*\%}\NormalTok{ r, }\AttributeTok{r =}\NormalTok{ rddi\_ols, }\AttributeTok{a =}\NormalTok{ A\_ols\_full)}

\NormalTok{ols\_absorbed }\OtherTok{\textless{}{-}} \FunctionTok{lm}\NormalTok{(y\_dd }\SpecialCharTok{\textasciitilde{}} \DecValTok{0} \SpecialCharTok{+}\NormalTok{ R\_dd\_ols, }\AttributeTok{data =}\NormalTok{ dat)}
\NormalTok{A\_ols\_absorb }\OtherTok{\textless{}{-}} \FunctionTok{attr}\NormalTok{(}\FunctionTok{vcovCR}\NormalTok{(ols\_absorbed, }\AttributeTok{type =} \StringTok{"CR2"}\NormalTok{, }\AttributeTok{cluster =}\NormalTok{ dat}\SpecialCharTok{$}\NormalTok{clust), }\StringTok{"adjustments"}\NormalTok{)}
\NormalTok{emat\_ols\_absorb }\OtherTok{\textless{}{-}} \FunctionTok{mapply}\NormalTok{(\textbackslash{}(r, a) a }\SpecialCharTok{\%*\%}\NormalTok{ r, }\AttributeTok{r =}\NormalTok{ rddi\_ols, }\AttributeTok{a =}\NormalTok{ A\_ols\_absorb)}
\CommentTok{\# si\_ols\_absorb \textless{}{-} mapply(\textbackslash{}(x,e) sum(x * e), x = emat\_ols\_absorb, e = split(residuals(ols\_absorbed), clust))}
\CommentTok{\# MR\_ols \textless{}{-} 1 / sum(sapply(rddi\_ols, \textbackslash{}(r) sum(r\^{}2)))}
\CommentTok{\# MR\_ols * si\_ols\_absorb}
\CommentTok{\# vcovCR(ols\_absorbed, type = "CR2", cluster = dat$clust, form = "estfun")}
\CommentTok{\# sum(si\_ols\_absorb\^{}2) * MR\_ols\^{}2}
\CommentTok{\# vcovCR(ols\_absorbed, type = "CR2", cluster = dat$clust)}
\CommentTok{\# vcovCR(ols\_fit, type = "CR2", cluster = dat$clust)["R","R"]}

\NormalTok{wls\_fit }\OtherTok{\textless{}{-}} \FunctionTok{lm}\NormalTok{(y }\SpecialCharTok{\textasciitilde{}}\NormalTok{ clust }\SpecialCharTok{+}\NormalTok{ R, }\AttributeTok{weights =}\NormalTok{ w, }\AttributeTok{data =}\NormalTok{ dat)}
\NormalTok{A\_wt\_full }\OtherTok{\textless{}{-}} \FunctionTok{attr}\NormalTok{(}\FunctionTok{vcovCR}\NormalTok{(wls\_fit, }\AttributeTok{type =} \StringTok{"CR2"}\NormalTok{, }\AttributeTok{cluster =}\NormalTok{ dat}\SpecialCharTok{$}\NormalTok{clust),}\StringTok{"adjustments"}\NormalTok{)}
\NormalTok{emat\_wt\_full }\OtherTok{\textless{}{-}} \FunctionTok{mapply}\NormalTok{(\textbackslash{}(r, w, a) a }\SpecialCharTok{\%*\%}\NormalTok{ (w }\SpecialCharTok{*}\NormalTok{ r), }\AttributeTok{r =}\NormalTok{ rddi\_wt, }\AttributeTok{w =}\NormalTok{ wi, }\AttributeTok{a =}\NormalTok{ A\_wt\_full)}
\CommentTok{\# si\_wt\_full \textless{}{-} mapply(\textbackslash{}(x,e) sum(x * e), x = emat\_wt\_full, e = split(residuals(wls\_fit), clust))}
\CommentTok{\# MR\_wt \textless{}{-} 1 / sum(mapply(\textbackslash{}(r,w) sum(w * r\^{}2), r = rddi\_wt, w = wi))}
\CommentTok{\# MR\_wt * sum(si\_wt\_full\^{}2) * MR\_wt}
\CommentTok{\# vcovCR(wls\_fit, type = "CR2", cluster = dat$clust)["R","R"]}

\NormalTok{wls\_absorbed }\OtherTok{\textless{}{-}} \FunctionTok{lm}\NormalTok{(y\_dd }\SpecialCharTok{\textasciitilde{}} \DecValTok{0} \SpecialCharTok{+}\NormalTok{ R\_dd\_wt, }\AttributeTok{weights =}\NormalTok{ w, }\AttributeTok{data =}\NormalTok{ dat)}
\NormalTok{A\_wt\_absorb }\OtherTok{\textless{}{-}} \FunctionTok{attr}\NormalTok{(}\FunctionTok{vcovCR}\NormalTok{(wls\_absorbed, }\AttributeTok{type =} \StringTok{"CR2"}\NormalTok{, }\AttributeTok{cluster =}\NormalTok{ dat}\SpecialCharTok{$}\NormalTok{clust), }\StringTok{"adjustments"}\NormalTok{)}
\NormalTok{emat\_wt\_absorb }\OtherTok{\textless{}{-}} \FunctionTok{mapply}\NormalTok{(\textbackslash{}(r, w, a) a }\SpecialCharTok{\%*\%}\NormalTok{ (w }\SpecialCharTok{*}\NormalTok{ r), }\AttributeTok{r =}\NormalTok{ rddi\_wt, }\AttributeTok{w =}\NormalTok{ wi, }\AttributeTok{a =}\NormalTok{ A\_wt\_absorb)}
\CommentTok{\# si\_wt\_absorb \textless{}{-} mapply(\textbackslash{}(x,e) sum(x * e), x = emat\_wt\_absorb, e = split(residuals(wls\_absorbed), clust))}
\CommentTok{\# MR\_wt * sum(si\_wt\_absorb\^{}2) * MR\_wt}
\CommentTok{\# vcovCR(wls\_absorbed, type = "CR2", cluster = dat$clust)}

\FunctionTok{data.frame}\NormalTok{(}
  \AttributeTok{ols\_full =} \FunctionTok{unlist}\NormalTok{(emat\_ols\_full),}
  \AttributeTok{ols\_absorb =} \FunctionTok{unlist}\NormalTok{(emat\_ols\_absorb),}
  \AttributeTok{wt\_full =} \FunctionTok{unlist}\NormalTok{(emat\_wt\_full),}
  \AttributeTok{wt\_absorb =} \FunctionTok{unlist}\NormalTok{(emat\_wt\_absorb)}
\NormalTok{)}
\end{Highlighting}
\end{Shaded}

\begin{verbatim}
##      ols_full ols_absorb   wt_full  wt_absorb
## 1   0.5735393  0.5735393  1.043413  1.0434125
## 2  -0.5735393 -0.5735393 -1.043413 -1.0434125
## 3   0.9375000  0.9375000  1.026980  2.3898897
## 4  -0.3125000 -0.3125000 -1.026980 -0.7734339
## 5  -0.3125000 -0.3125000 -1.026980 -0.7734339
## 6  -0.3125000 -0.3125000 -1.026980 -0.7734339
## 7   1.0758287  1.0758287  1.150447  4.5774116
## 8  -0.2151657 -0.2151657 -1.150447 -1.0015292
## 9  -0.2151657 -0.2151657 -1.150447 -1.0015292
## 10 -0.2151657 -0.2151657 -1.150447 -1.0015292
## 11 -0.2151657 -0.2151657 -1.150447 -1.0015292
## 12 -0.2151657 -0.2151657 -1.150447 -1.0015292
\end{verbatim}

\hypertarget{a-revised-theorem-2}{%
\section{A revised Theorem 2}\label{a-revised-theorem-2}}

The implication of the original Theorem 2 was that using the modified
adjustment matrices \(\tilde{\mathbf{A}}_i\) to calculate the CR2
estimator yields the same result as using the full adjustment matrices
\(\mathbf{A}_i\). Although this does not hold under the general
conditions given above, a modified version of the theorem does hold for
the more limited case of ordinary (unweighted) least squares regression
with an ``independence'' working model. The precise conditions are given
in the following theorem.

\begin{thm}
\label{thm:absorb}
Let $\mathbf{L}_i = \left(\mathbf{\ddot{U}}'\mathbf{\ddot{U}} - \mathbf{\ddot{U}}_i'\mathbf{\ddot{U}}_i\right)$ and assume that $\mathbf{L}_1,...,\mathbf{L}_m$ have full rank $r + s$. If $\mathbf{W}_i = \mathbf{I}_i$ and $\boldsymbol\Phi_i = \mathbf{I}_i$ for $i = 1,...,m$ and $\mathbf{T}_i \mathbf{T}_k' = \mathbf{0}$ for $i \neq k$, then $\mathbf{A}_i \mathbf{\ddot{R}}_i = \mathbf{\tilde{A}}_i \mathbf{\ddot{R}}_i$, where $\mathbf{A}_i$ and $\tilde{\mathbf{A}}_i$ are as defined in (\ref{eq:A-matrix}) and (\ref{eq:A-modified}), respectively.
\end{thm}

The implication of the revised theorem is that, for ordinary least
squares regression with an ``independence'' working model, calculating
the CR2 with the modified adjustment matrices \(\tilde{\mathbf{A}}_i\)
leads to the same result as using the full adjustment matrices
\(\mathbf{A}_i\). The equality does not hold for weighted or generalized
least squares, nor for ordinary least squares with working models other
than \(\boldsymbol\Phi_i = \mathbf{I}_i\).

\hypertarget{proof}{%
\subsection{Proof}\label{proof}}

Setting \(\boldsymbol\Phi_i = \mathbf{I}_i\) and observing that
\(\mathbf{\ddot{U}}_i'\mathbf{T}_i = \mathbf{0}\) for \(i = 1,...,m\),
it follows that \begin{align}
\mathbf{B}_i &= \mathbf{D}_i \mathbf{C}_i \left(\mathbf{I} - \mathbf{H_{\ddot{U}}}\right) \left(\mathbf{I} - \mathbf{H_T}\right) \boldsymbol\Phi \left(\mathbf{I} - \mathbf{H_T}\right)' \left(\mathbf{I} - \mathbf{H_{\ddot{U}}}\right)' \mathbf{C}_i' \mathbf{D}_i' \nonumber \\ 
&= \mathbf{C}_i \left(\mathbf{I} - \mathbf{H_{\ddot{U}}} - \mathbf{H_T}\right) \left(\mathbf{I} - \mathbf{H_{\ddot{U}}} - \mathbf{H_T}\right) \mathbf{C}_i' \nonumber\\ 
\label{eq:B_i}
&= \left(\mathbf{I}_i - \mathbf{\ddot{U}}_i \mathbf{M_{\ddot{U}}}\mathbf{\ddot{U}}_i' - \mathbf{T}_i \mathbf{M_T}\mathbf{T}_i'\right)
\end{align} and similarly, \begin{equation}
\label{eq:Btilde_i}
\tilde{\mathbf{B}}_i = \left(\mathbf{I}_i - \mathbf{\ddot{U}}_i \mathbf{M_{\ddot{U}}}\mathbf{\ddot{U}}_i'\right).
\end{equation}

We now show that \(\tilde{\mathbf{A}}_i \mathbf{T}_i = \mathbf{T}_i\).
Denote the rank of \(\mathbf{\ddot{U}}_i\) as
\(u_i \leq \min \left\{n_i, r + s \right\}\) and take the thin QR
decomposition of \(\mathbf{\ddot{U}}_i\) as
\(\mathbf{\ddot{U}}_i = \mathbf{Q}_i \mathbf{R}_i\), where
\(\mathbf{Q}_i\) is an \(n_i \times u_i\) semi-orthonormal matrix and
\(\mathbf{R}_i\) is a \(u_i \times r + s\) matrix of rank \(u_i\), with
\(\mathbf{Q}_i'\mathbf{Q}_i = \mathbf{I}\). Note that
\(\mathbf{Q}_i'\mathbf{T}_i = \mathbf{0}\). From the observation that
\(\tilde{\mathbf{B}}_i\) can be written as \[
\tilde{\mathbf{B}}_i = \mathbf{I}_i - \mathbf{Q}_i \mathbf{Q}_i' + \mathbf{Q}_i \left(\mathbf{I} - \mathbf{R}_i \mathbf{M}_{\mathbf{\ddot{U}}} \mathbf{R}_i'\right)\mathbf{Q}_i',
\] it can be seen that \begin{equation}
\tilde{\mathbf{A}}_i = \tilde{\mathbf{B}}_i^{+1/2} = \mathbf{I}_i - \mathbf{Q}_i \mathbf{Q}_i' + \mathbf{Q}_i \left(\mathbf{I} - \mathbf{R}_i \mathbf{M}_{\mathbf{\ddot{U}}} \mathbf{R}_i'\right)^{+1/2} \mathbf{Q}_i'.
\end{equation} It follows that
\(\tilde{\mathbf{A}}_i \mathbf{T}_i = \mathbf{T}_i\).

Setting \begin{equation}
\mathbf{A}_i = \tilde{\mathbf{A}}_i - \mathbf{T}_i \mathbf{M_T}\mathbf{T}_i',
\end{equation} observe that \begin{align*}
\mathbf{B}_i \mathbf{A}_i \mathbf{B}_i \mathbf{A}_i &= \left(\tilde{\mathbf{B}}_i - \mathbf{T}_i \mathbf{M_T}\mathbf{T}_i'\right) \left(\tilde{\mathbf{A}}_i - \mathbf{T}_i \mathbf{M_T}\mathbf{T}_i'\right)\left(\tilde{\mathbf{B}}_i - \mathbf{T}_i \mathbf{M_T}\mathbf{T}_i'\right) \left(\tilde{\mathbf{A}}_i - \mathbf{T}_i \mathbf{M_T}\mathbf{T}_i'\right) \\
&= \left(\tilde{\mathbf{B}}_i\tilde{\mathbf{A}}_i - \mathbf{T}_i \mathbf{M_T}\mathbf{T}_i'\right)\left(\tilde{\mathbf{B}}_i\tilde{\mathbf{A}}_i - \mathbf{T}_i \mathbf{M_T}\mathbf{T}_i'\right) \\
&= \left(\tilde{\mathbf{B}}_i\tilde{\mathbf{A}}_i\tilde{\mathbf{B}}_i\tilde{\mathbf{A}}_i - \mathbf{T}_i \mathbf{M_T}\mathbf{T}_i'\right) \\
&= \left(\tilde{\mathbf{B}}_i - \mathbf{T}_i \mathbf{M_T}\mathbf{T}_i'\right) \\
&= \mathbf{B}_i.
\end{align*} It follows that \(\mathbf{A}_i\) is the symmetric square
root of the Moore-Penrose inverse of \(\mathbf{B}_i\), i.e.,
\(\mathbf{A}_i = \mathbf{B}_i^{+1/2}\). Finally, because
\(\mathbf{T}_i ' \mathbf{\ddot{R}}_i= \mathbf{0}\), it can be seen that
\(\mathbf{A}_i \mathbf{\ddot{R}}_i = \left(\tilde{\mathbf{A}}_i - \mathbf{T}_i \mathbf{M_T}\mathbf{T}_i'\right)\mathbf{\ddot{R}}_i = \tilde{\mathbf{A}}_i \mathbf{\ddot{R}}_i\).

\hypertarget{references}{%
\section*{References}\label{references}}
\addcontentsline{toc}{section}{References}

\hypertarget{refs}{}
\begin{CSLReferences}{1}{0}
\leavevmode\vadjust pre{\hypertarget{ref-Bell2002bias}{}}%
Bell, Robert M, and Daniel F McCaffrey. 2002. {``{Bias reduction in
standard errors for linear regression with multi-stage samples}.''}
\emph{Survey Methodology} 28 (2): 169--81.

\leavevmode\vadjust pre{\hypertarget{ref-McCaffrey2001generalizations}{}}%
McCaffrey, Daniel F, Robert M Bell, and Carsten H Botts. 2001.
{``{Generalizations of biased reduced linearization}.''} In
\emph{Proceedings of the Annual Meeting of the American Statistical
Association}. 1994.

\leavevmode\vadjust pre{\hypertarget{ref-pustejovsky2018small}{}}%
Pustejovsky, James E, and Elizabeth Tipton. 2018. {``Small-Sample
Methods for Cluster-Robust Variance Estimation and Hypothesis Testing in
Fixed Effects Models.''} \emph{Journal of Business \& Economic
Statistics} 36 (4): 672--83.
\url{https://doi.org/10.1080/07350015.2016.1247004}.

\end{CSLReferences}

\bibliographystyle{agsm}
\bibliography{bibliography.bib}


\end{document}
