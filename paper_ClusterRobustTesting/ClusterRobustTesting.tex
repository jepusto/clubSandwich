\documentclass[man, noextraspace, floatsintext]{apa6}\usepackage[]{graphicx}\usepackage[]{color}
%% maxwidth is the original width if it is less than linewidth
%% otherwise use linewidth (to make sure the graphics do not exceed the margin)
\makeatletter
\def\maxwidth{ %
  \ifdim\Gin@nat@width>\linewidth
    \linewidth
  \else
    \Gin@nat@width
  \fi
}
\makeatother

\definecolor{fgcolor}{rgb}{0.345, 0.345, 0.345}
\newcommand{\hlnum}[1]{\textcolor[rgb]{0.686,0.059,0.569}{#1}}%
\newcommand{\hlstr}[1]{\textcolor[rgb]{0.192,0.494,0.8}{#1}}%
\newcommand{\hlcom}[1]{\textcolor[rgb]{0.678,0.584,0.686}{\textit{#1}}}%
\newcommand{\hlopt}[1]{\textcolor[rgb]{0,0,0}{#1}}%
\newcommand{\hlstd}[1]{\textcolor[rgb]{0.345,0.345,0.345}{#1}}%
\newcommand{\hlkwa}[1]{\textcolor[rgb]{0.161,0.373,0.58}{\textbf{#1}}}%
\newcommand{\hlkwb}[1]{\textcolor[rgb]{0.69,0.353,0.396}{#1}}%
\newcommand{\hlkwc}[1]{\textcolor[rgb]{0.333,0.667,0.333}{#1}}%
\newcommand{\hlkwd}[1]{\textcolor[rgb]{0.737,0.353,0.396}{\textbf{#1}}}%

\usepackage{framed}
\makeatletter
\newenvironment{kframe}{%
 \def\at@end@of@kframe{}%
 \ifinner\ifhmode%
  \def\at@end@of@kframe{\end{minipage}}%
  \begin{minipage}{\columnwidth}%
 \fi\fi%
 \def\FrameCommand##1{\hskip\@totalleftmargin \hskip-\fboxsep
 \colorbox{shadecolor}{##1}\hskip-\fboxsep
     % There is no \\@totalrightmargin, so:
     \hskip-\linewidth \hskip-\@totalleftmargin \hskip\columnwidth}%
 \MakeFramed {\advance\hsize-\width
   \@totalleftmargin\z@ \linewidth\hsize
   \@setminipage}}%
 {\par\unskip\endMakeFramed%
 \at@end@of@kframe}
\makeatother

\definecolor{shadecolor}{rgb}{.97, .97, .97}
\definecolor{messagecolor}{rgb}{0, 0, 0}
\definecolor{warningcolor}{rgb}{1, 0, 1}
\definecolor{errorcolor}{rgb}{1, 0, 0}
\newenvironment{knitrout}{}{} % an empty environment to be redefined in TeX

\usepackage{alltt}
\newcommand{\bibfile}{bibliography}

\usepackage[natbibapa]{apacite}

\raggedbottom

\usepackage{amssymb}
\usepackage{amsmath}
\usepackage{amsthm}

\usepackage{graphicx}

\usepackage{fixltx2e}
\usepackage{subcaption}
\usepackage{float}

%\usepackage{array}
%\usepackage{multirow}
%\usepackage{rotating}
%\setlength{\rotFPtop}{0pt plus 1fil}
%\usepackage[draft]{changes}
%\geometry{twoside=false, top=1in, bottom=1in, left=1in, right=1in}
%\usepackage[textwidth=1in, textsize=tiny]{todonotes}

\newcommand{\Prob}{\text{Pr}}
\newcommand{\E}{\text{E}}
\newcommand{\Cov}{\text{Cov}}
\newcommand{\corr}{\text{corr}}
\newcommand{\Var}{\text{Var}}
\newcommand{\iid}{\stackrel{\text{iid}}{\sim}}
\newcommand{\tr}{\text{tr}}
\newcommand{\bM}{\mathbf}
\newcommand{\bS}{\boldsymbol}


\title{Small sample correction methods for cluster-robust variance estimators and hypothesis tests}
\shorttitle{SMALL SAMPLE CLUSTER-ROBUST METHODS}
\twoauthors{James E. Pustejovsky}{Elizabeth Tipton}
\twoaffiliations{The University of Texas at Austin}{Columbia University}
\leftheader{}

\abstract{}

\keywords{}

\authornote{}
\IfFileExists{upquote.sty}{\usepackage{upquote}}{}
\begin{document}


\maketitle

Cluster-robust variance estimators (CRVE) and hypothesis tests based upon such estimators are ubiquitous in applied econometric work. Nearly every respectable paper in the past 15 years uses cluster-robust variance estimators because to do otherwise would be to risk being seen as insufficiently rigorous (or worse, anti-conservative....ughh....how gauche!).

There's been a lot of fretting recently that even CRVE may actually not be rigorous enough. Cite the following people so as not to get their ire up:
\begin{itemize}
\item \citet{Brewer2013inference}
\item \citet{Cameron2008bootstrap}
\item \citet{Cameron2015practitioners}
\item \citet{Carter2013asymptotic}
\item \citet{Ibragimov2010tstatistic}
\item \citet{Imbens2012robust}
\item \citet{Kezdi2004robust}
\item \citet{McCaffrey2001generalizations}
\item \citet{McCaffrey2006improved}
\item \citet{Webb2013wild}
\item \citet{Kline2012score}
\end{itemize}

\subsection{Econometric framework}

We will consider linear regression models in which the errors within a cluster have an unknown variance structure. The model is
\begin{equation}
\label{eq:model_vector}
\bM{Y}_j = \bM{X}_j \bS\beta + \bM{e}_j,
\end{equation}
for $j=1,...,m$, where $\bM{Y}_j$ is $n_j \times 1$, $\bM{X}_j$ is an $n_j \times p$ matrix of regressors for cluster $j$, $\bS\beta$ is a $p \times 1$ vector, and $\bM{e}_j$ is an $n_j \times 1$ vector of errors. Assume that $\E\left(\bM{e}_j\left|\bM{X}_j\right.\right) = \bM{0}$ and $\Var\left(\bM{e}_j\left|\bM{X}_j\right.\right) = \bS\Sigma_j$, for $j = 1,...,m$, where $\bS\Sigma_1,...,\bS\Sigma_m$ may be unknown. Let $\bS\Sigma = \bigoplus_{j=1}^m \bS\Sigma_j$.

The vector of regression coefficients is estimated by weighted least squares (WLS). Given a set of $m$ weighting matrices $\bM{W}_1,...,\bM{W}_m$, the WLS estimator is 
\begin{equation}
\label{eq:WLS}
\bS{\hat\beta} = \bM{M} \sum_{j=1}^m \bM{X}_j' \bM{W}_j \bM{Y}_j, 
\end{equation}
where $\bM{M} = \left(\sum_{j=1}^m \bM{X}_j' \bM{W}_j \bm{X}_j\right)^{-1}$.

Common choices for weighting include the unweighted case, in which $\bM{W}_j$ is an identity matrix of dimension $n_j \times n_j$, and inverse-variance weighting under a working model. In the latter case, the errors are assumed to follow some known structure, $\Var\left(\bM{e}_j\left|\bM{X}_j\right.\right) = \bS\Phi_j$, where $\bS\Phi_j$ is either known or a function of a low-dimensional parameter. The weighting matrices are then taken to be $\bM{W}_j = \bS\Phi_j^{-1}$, possibly based on estimates of the variance parameters. 

The WLS estimator also encompasses the estimator proposed by \citet{Ibragimov2010tstatistic} for clustered data. Assuming that $\bM{X}_j$ has rank $p$, their proposed approach involves estimating $\bS\beta$ separately within each cluster and taking the simple average of these estimates. The resulting average is equivalent to the WLS estimator with weights $\bM{W}_j = \bM{X}_j \left(\bM{X}_j'\bM{X}_j\right)^{-2} \bM{X}_j$.

\section{Cluster-robust variance estimators}

\subsection{Considerations with panel models}

\section{Hypothesis testing}

\subsection{Single-constraint tests}

\subsection{Multiple-constraint tests}

\section{Examples}

\section{Simulation evidence}

\section{Discussion}

\bibliographystyle{apacite}
\bibliography{\bibfile}


\end{document}
