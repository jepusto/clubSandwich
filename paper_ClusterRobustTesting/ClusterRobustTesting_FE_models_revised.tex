% !TeX program = pdfLaTeX
\documentclass[12pt]{article}
\usepackage{amsmath}
\usepackage{graphicx,psfrag,epsf}
\usepackage{enumerate}
\usepackage{natbib}
\usepackage{textcomp}
\usepackage[hyphens]{url} % not crucial - just used below for the URL
\usepackage{hyperref}

%\pdfminorversion=4
% NOTE: To produce blinded version, replace "0" with "1" below.
\newcommand{\blind}{0}

% DON'T change margins - should be 1 inch all around.
\addtolength{\oddsidemargin}{-.5in}%
\addtolength{\evensidemargin}{-.5in}%
\addtolength{\textwidth}{1in}%
\addtolength{\textheight}{1.3in}%
\addtolength{\topmargin}{-.8in}%

%% load any required packages here



% tightlist command for lists without linebreak
\providecommand{\tightlist}{%
  \setlength{\itemsep}{0pt}\setlength{\parskip}{0pt}}



\usepackage{amsthm}
\newtheorem{thm}{Theorem}
\newtheorem{lem}{Lemma}
\usepackage{booktabs}
\usepackage{longtable}
\usepackage{float}    % for fig.pos='H'
\usepackage{rotfloat} % for sidewaysfigure
\hypersetup{hidelinks}
\newcommand{\Prob}{\text{Pr}}
\newcommand{\E}{\text{E}}
\newcommand{\Cov}{\text{Cov}}
\newcommand{\corr}{\text{corr}}
\newcommand{\Var}{\text{Var}}
\newcommand{\iid}{\stackrel{\text{iid}}{\sim}}
\newcommand{\tr}{\text{tr}}
\newcommand{\bm}{\mathbf}
\newcommand{\bs}{\boldsymbol}

\begin{document}


\def\spacingset#1{\renewcommand{\baselinestretch}%
{#1}\small\normalsize} \spacingset{1}


%%%%%%%%%%%%%%%%%%%%%%%%%%%%%%%%%%%%%%%%%%%%%%%%%%%%%%%%%%%%%%%%%%%%%%%%%%%%%%

\if0\blind
{
  \title{\bf Small sample methods for cluster-robust variance estimation
and hypothesis testing in fixed effects models}

  \author{
        James E. Pustejovsky \thanks{Department of Educational
Psychology, University of Wisconsin - Madison, 1025 West Johnson Street,
Madison, WI 53706. Email:
\href{mailto:pustejovsky@wisc.edu}{\nolinkurl{pustejovsky@wisc.edu}}} \\
    University of Wisconsin - Madison\\
     and \\     Elizabeth Tipton \thanks{Department of Statistics,
Northwestern University. Email:
\href{mailto:tipton@northwestern.edu}{\nolinkurl{tipton@northwestern.edu}}} \\
    Northwestern University\\
      }
  \maketitle
} \fi

\if1\blind
{
  \bigskip
  \bigskip
  \bigskip
  \begin{center}
    {\LARGE\bf Small sample methods for cluster-robust variance
estimation and hypothesis testing in fixed effects models}
  \end{center}
  \medskip
} \fi

\bigskip
\begin{abstract}
In panel data models and other regressions with unobserved effects,
fixed effects estimation is often paired with cluster-robust variance
estimation (CRVE) in order to account for heteroskedasticity and
un-modeled dependence among the errors. Although asymptotically
consistent, CRVE can be biased downward when the number of clusters is
small, leading to hypothesis tests with rejection rates that are too
high. More accurate tests can be constructed using bias-reduced
linearization (BRL), which corrects the CRVE based on a working model,
in conjunction with a Satterthwaite approximation for t-tests. We
propose a generalization of BRL that can be applied in models with
arbitrary sets of fixed effects, where the original BRL method is
undefined, and describe how to apply the method when the regression is
estimated after absorbing the fixed effects. We also propose a
small-sample test for multiple-parameter hypotheses, which generalizes
the Satterthwaite approximation for t-tests. In simulations covering a
wide range of scenarios, we find that the conventional cluster-robust
Wald test can severely over-reject while the proposed small-sample test
maintains Type I error close to nominal levels. The proposed methods are
implemented in an R package called clubSandwich.
\end{abstract}

\noindent%
{\it Keywords:} cluster dependence, fixed effects, robust standard
errors, small samples
\vfill

\newpage
\spacingset{1.45} % DON'T change the spacing!

\hypertarget{sec:intro}{%
\section{INTRODUCTION}\label{sec:intro}}

In many economic analyses, interest centers on the parameters of a
linear regression model estimated by ordinary or weighted least squares
from data exhibiting within-group dependence. Such dependence can arise
from sampling or random assignment of aggregate units (e.g., counties,
districts, villages), each of which contains multiple observations; from
repeated measurement of an outcome on a common set of units, as in panel
data; or from model misspecification, as in analysis of regression
discontinuity designs \citep[e.g.,][]{Lee2008regression}. A common
approach to inference in these settings is to use a cluster-robust
variance estimator
\citep[CRVE,][]{Arellano1987computing, Liang1986longitudinal, white1984asymptotic}.
The advantage of the CRVE is that it produces consistent standard errors
and test statistics without imposing strong parametric assumptions about
the correlation structure of the errors in the model. Instead, the
method relies on the weaker assumption that units can be grouped into
clusters that are mutually independent. In the past decade, use of CRVEs
has become standard practice for micro-economic researchers, as
evidenced by coverage in major textbooks and review articles
\citep[e.g.,][]{Wooldridge2010econometric, Angrist2009mostly, Cameron2015practitioners}.

As a leading example, consider a difference-in-differences analysis of
state-by-year panel data, where the goal is to understand the effects on
employment outcomes of several state-level policy shifts. Each policy
effect would be parameterized as a dummy variable in a regression model,
which might also include other demographic controls. It is also common
to include fixed effects for states and time-points in order to control
for unobserved confounding in each dimension. The model could be
estimated by least squares with the fixed effects included as dummy
variables (or what we will call the LSDV estimator). More commonly, the
effects of the policy indicators would be estimated after absorbing the
fixed effects, a computational technique that is also known as the fixed
effects estimator or ``within transformation''
\citep{Wooldridge2010econometric}. Standard errors would then be
clustered by state to account for residual dependence in the errors from
a given state, and these clustered standard errors would be used to test
hypotheses about the set of policies. The need to cluster the standard
errors by state, even when including state fixed effects, was
highlighted by \citet{Bertrand2004how}, who showed that to do otherwise
can lead to inappropriately small standard errors and hypothesis tests
with incorrect rejection rates.

The consistency property of CRVEs is asymptotic in the number of
independent clusters \citep{Wooldridge2003cluster}. Recent
methodological work has demonstrated that CRVEs can be biased downward
and associated hypothesis tests can have Type-I error rates considerably
in excess of nominal levels when based on a small or moderate number of
clusters \citep[e.g.,][]{MacKinnon2016wild}.
\citet{Cameron2015practitioners} provided an extensive review of this
literature, including a discussion of current practice, possible
solutions, and open problems. In particular, they demonstrated that
small-sample corrections for t-tests implemented in common software
packages such as Stata and SAS do not provide adequate control of Type-I
error.

\citet{Bell2002bias} proposed a method that improves the small-sample
properties of CRVEs \citep[see also][]{McCaffrey2001generalizations}.
The method, called bias-reduced linearization (BRL), entails adjusting
the CRVE so that it is exactly unbiased under a working model specified
by the analyst, while also remaining asymptotically consistent under
arbitrary true variance structures. Simulations reported by
\citet{Bell2002bias} demonstrate that the BRL correction serves to
reduce the bias of the CRVE even when the working model is misspecified.
The same authors also proposed small-sample corrections to
single-parameter hypothesis tests using the BRL variance estimator,
based on Satterthwaite \citep{Bell2002bias} or saddlepoint
approximations \citep{McCaffrey2006improved}. In an analysis of a
longitudinal cluster-randomized trial with 35 clusters,
\citet{Angrist2009effects} observed that the BRL correction makes a
difference for inferences.

Despite a growing body of evidence that BRL performs well
\citep[e.g.,][]{Imbens2015robust}, several problems with the method
hinder its wider application. First, \citet{Angrist2009mostly} noted
that the BRL correction is undefined in some commonly used models, such
as state-by-year panels that include fixed effects for states and for
years \citep[see also][]{Young2016improved}. Second, in models with
fixed effects, the magnitude of the BRL adjustment depends on whether it
is computed based on the full design matrix (i.e., the LSDV estimator)
or after absorbing the fixed effects. Third, extant methods for
hypothesis testing based on BRL are limited to single-parameter
constraints \citep{Bell2002bias, McCaffrey2006improved} and small-sample
methods for multiple-parameter hypothesis tests remain lacking.

This paper addresses each of these concerns in turn, with the aim of
extending the BRL method so that is suitable for general application.
First, we describe a simple modification of the BRL adjustment that
remains well-defined in models with arbitrary sets of fixed effects,
where existing BRL adjustments break down. Second, we demonstrate how to
calculate the BRL adjustments based on the fixed effects estimator and
identify conditions under which first-stage absorption of the fixed
effects can be ignored. Finally, we propose a procedure for testing
multiple-parameter hypotheses by approximating the sampling distribution
of the Wald statistic using Hotelling's \(T^2\) distribution with
estimated degrees of freedom. The method is a generalization of the
Satterthwaite correction proposed by \citet{Bell2002bias} for single
parameter constraints. The proposed methods are implemented in the R
package \texttt{clubSandwich}, which is available on the Comprehensive R
Archive Network.

Our work is related to a stream of recent literature that has examined
methods for cluster-robust inference with a small number of clusters.
\citet{Conley2011inference} proposed methods for hypothesis testing in a
difference-in-differences setting where the number of treated units is
small and fixed, while the number of untreated units increases
asymptotically. Ibragimov and Müller
\citetext{\citeyear{Ibragimov2010tstatistic}; \citeyear{Ibragimov2016inference}}
proposed cluster-robust t-tests that maintain the nominal Type-I error
rate by re-weighting within-cluster estimators of the target parameter.
\citet{Young2016improved} proposed a Satterthwaite correction for
t-tests based on a different type of bias correction to the CRVE, where
the bias correction term is derived under a working model.
\citet{Cameron2008bootstrap} investigated a range of bootstrapping
procedures that provide improved Type-I error control in small samples,
finding that a cluster wild-bootstrap technique was particularly
accurate in small samples. Nearly all of this work has focused on
single-parameter hypothesis tests only. For multiple-parameter
constraints, \citet{Cameron2015practitioners} suggested an ad hoc
degrees of freedom adjustment and noted, as an alternative, that
bootstrapping techniques can in principle be applied to
multiple-parameter tests. However, little methodological work has
examined the accuracy of multiple-parameter tests.

The paper is organized as follows. The remainder of this section
introduces our econometric framework and reviews standard CRVE methods,
as implemented in most software applications. Section \ref{sec:BRL}
reviews the original BRL correction and describes modifications that
make it possible to implement BRL in a broad class of models with fixed
effects. Section \ref{sec:testing} discusses hypothesis tests based on
the BRL-adjusted CRVE. Section \ref{sec:simulation} reports a simulation
study examining the null rejection rates of multiple-parameter
hypothesis tests, where we find that the small-sample test offers
drastic improvements over commonly implemented alternatives. Section
\ref{sec:examples} illustrates the use of the proposed hypothesis tests
in two applications. Section \ref{sec:conclusion} concludes and
discusses avenues for future work.

\hypertarget{econometric-framework}{%
\subsection{Econometric framework}\label{econometric-framework}}

We consider a linear regression model of the form, \begin{equation}
\label{eq:fixed_effects}
\mathbf{y}_i = \mathbf{R}_i \boldsymbol\beta + \mathbf{S}_i \boldsymbol\gamma + \mathbf{T}_i \boldsymbol\mu + \boldsymbol\epsilon_i,
\end{equation} where there are a total of \(m\) clusters; cluster \(i\)
contains \(n_i\) units; \(\mathbf{y}_i\) is a vector of the \(n_i\)
values of the outcome for units in cluster \(i\); \(\mathbf{R}_i\) is an
\(n_i \times r\) matrix containing predictors of primary interest (e.g.,
policy variables) and any additional controls; \(\mathbf{S}_i\) is an
\(n_i \times s\) matrix describing fixed effects that are identified
across multiple clusters; and \(\mathbf{T}_i\) is an \(n_i \times t\)
matrix describing cluster-specific fixed effects, which must satisfy
\(\mathbf{T}_h \mathbf{T}_i' = \mathbf{0}\) for \(h \neq i\). Note that
the distinction between the covariates \(\mathbf{R}_i\) versus the fixed
effects \(\mathbf{S}_i\) is arbitrary and depends on the analyst's
inferential goals. In a fixed effects model for state-by-year panel
data, \(\mathbf{R}_i\) would include variables describing policy
changes, as well as additional demographic controls; \(\mathbf{S}_i\)
would include year fixed effects; and \(\mathbf{T}_i\) would indicate
state fixed effects (and perhaps also state-specific time trends).
Interest would center on testing hypotheses regarding the coefficients
in \(\boldsymbol\beta\) that correspond to the policy indicators, while
\(\boldsymbol\gamma\) and \(\boldsymbol\mu\) would be treated as
incidental.

We shall assume that
\(\text{E}\left(\boldsymbol\epsilon_i\left|\mathbf{R}_i,\mathbf{S}_i, \mathbf{T}_i\right.\right) = \mathbf{0}\)
and
\(\text{Var}\left(\boldsymbol\epsilon_i\left|\mathbf{R}_i,\mathbf{S}_i,\mathbf{T}_i\right.\right) = \boldsymbol\Sigma_i\),
for \(i = 1,...,m\), where the form of
\(\boldsymbol\Sigma_1,...,\boldsymbol\Sigma_m\) may be unknown but the
errors are independent across clusters. Let
\(\mathbf{U}_i = \left[\mathbf{R}_i \ \mathbf{S}_i \right]\) denote the
set of predictors that are identified across multiple clusters,
\(\mathbf{X}_i = \left[\mathbf{R}_i \ \mathbf{S}_i \ \mathbf{T}_i \right]\)
denote the full set of predictors,
\(\boldsymbol\alpha = \left(\boldsymbol\beta', \boldsymbol\gamma', \boldsymbol\mu' \right)'\),
and \(p = r + s + t\). Let \(N = \sum_{i=1}^m n_i\) denote the total
number of observations. Let \(\mathbf{y}\), \(\mathbf{R}\),
\(\mathbf{S}\), \(\mathbf{T}\), \(\mathbf{U}\), \(\mathbf{X}\), and
\(\boldsymbol\epsilon\) denote the matrices obtained by stacking their
corresponding components, as in
\(\mathbf{R} = \left(\mathbf{R}_1' \ \mathbf{R}_2' \ \cdots \ \mathbf{R}_m'\right)'\).

We assume that \(\boldsymbol\beta\) is estimated by weighted least
squares (WLS) using symmetric, full rank weighting matrices
\(\mathbf{W}_1,...,\mathbf{W}_m\). Clearly, the WLS estimator includes
ordinary least squares (OLS) as a special case. More generally, the WLS
estimator encompasses feasible generalized least squares, where it is
assumed that
\(\text{Var}\left(\mathbf{e}_i\left|\mathbf{X}_i\right.\right) = \boldsymbol\Phi_i\),
a known function of a low-dimensional parameter. For example, an
auto-regressive error structure might be posited to describe repeated
measures on an individual over time. The weighting matrices are then
taken to be \(\mathbf{W}_i = \hat{\boldsymbol\Phi}_i^{-1}\), where the
\(\hat{\boldsymbol\Phi}_i\) are constructed from estimates of the
variance parameter. Finally, for analysis of data from complex survey
designs, WLS may be used with sampling weights in order to account for
unequal selection probabilities.

\hypertarget{absorption}{%
\subsection{Absorption}\label{absorption}}

The goal of most analyses is to estimate and test hypotheses regarding
the parameters in \(\boldsymbol\beta\), while the fixed effects
\(\boldsymbol\gamma\) and \(\boldsymbol\mu\) are not of inferential
interest. Furthermore, LSDV estimation becomes computationally intensive
and numerically inaccurate if the model includes a large number of fixed
effects (i.e., \(s + t\) large). A commonly implemented alternative to
LSDV is to first absorb the fixed effects, which leaves only the \(r\)
parameters in \(\boldsymbol\beta\) to be estimated. Because Section
\ref{sec:BRL} examines the implications of absorption for application of
the BRL adjustment, we now formalize this procedure. Denote the full
block-diagonal weighting matrix as
\(\mathbf{W} = \text{diag}\left(\mathbf{W}_1,...,\mathbf{W}_m\right)\).
Let \(\mathbf{K}\) be the \(p \times r\) matrix that selects the
covariates of interest, so that \(\mathbf{X} \mathbf{K} = \mathbf{R}\)
and \(\mathbf{K}'\boldsymbol\alpha = \boldsymbol\beta\). For a generic
matrix \(\mathbf{Z}\) of full column rank, let
\(\mathbf{M_Z} = \left(\mathbf{Z}'\mathbf{W}\mathbf{Z}\right)^{-1}\) and
\(\mathbf{H_Z} = \mathbf{Z}\mathbf{M_Z}\mathbf{Z}'\mathbf{W}\).

The absorption technique involves obtaining the residuals from the
regression of \(\mathbf{y}\) on \(\mathbf{T}\) and from the multivariate
regression of \([\mathbf{R} \ \mathbf{S}]\) on \(\mathbf{T}\). The
\(\mathbf{y}\) residuals and \(\mathbf{R}\) residuals are then regressed
on the \(\mathbf{S}\) residuals. Finally, these twice-regressed
\(\mathbf{y}\) residuals are regressed on the twice-regressed
\(\mathbf{R}\) residuals to obtain the WLS estimates of
\(\boldsymbol\beta\). Let
\(\mathbf{\ddot{S}} = \left(\mathbf{I} - \mathbf{H_T}\right)\mathbf{S}\),
\(\mathbf{\ddot{R}} = \left(\mathbf{I} - \mathbf{H_{\ddot{S}}}\right)\left(\mathbf{I} - \mathbf{H_T}\right)\mathbf{R}\),
and
\(\mathbf{\ddot{y}} = \left(\mathbf{I} - \mathbf{H_{\ddot{S}}}\right)\left(\mathbf{I} - \mathbf{H_T}\right)\mathbf{y}\).
In what follows, subscripts on \(\mathbf{\ddot{R}}\),
\(\mathbf{\ddot{S}}\), \(\mathbf{\ddot{U}}\), and \(\mathbf{\ddot{y}}\)
refer to the rows of these matrices corresponding to a specific cluster.
The WLS estimator of \(\boldsymbol\beta\) can then be written as
\begin{equation}
\label{eq:WLS}
\boldsymbol{\hat\beta} = \mathbf{M_{\ddot{R}}} \sum_{i=1}^m \mathbf{\ddot{R}}_i' \mathbf{W}_i \mathbf{\ddot{y}}_i. 
\end{equation} This estimator is algebraically identical to the LSDV
estimator,
\(\boldsymbol{\hat\beta} = \mathbf{K}'\mathbf{M_X} \mathbf{X}' \mathbf{W} \mathbf{y}\),
but avoids the need to solve a system of \(p\) linear equations. For
further details on sequential absorption, see
\citet{Davis2002estimating}. In the remainder, we assume that fixed
effects are absorbed before estimation of \(\boldsymbol\beta\).

\hypertarget{standard-crve}{%
\subsection{Standard CRVE}\label{standard-crve}}

The WLS estimator \(\boldsymbol{\hat\beta}\), has true variance
\begin{equation}
\label{eq:var_WLS}
\text{Var}\left(\boldsymbol{\hat\beta}\right) = \mathbf{M_{\ddot{R}}}\left(\sum_{i=1}^m \mathbf{\ddot{R}}_i' \mathbf{W}_i \boldsymbol\Sigma_i \mathbf{W}_i\mathbf{\ddot{R}}_i\right) \mathbf{M_{\ddot{R}}},
\end{equation} which depends upon the unknown variance matrices
\(\boldsymbol\Sigma_i\).

The CRVE involves estimating
\(\text{Var}\left(\boldsymbol{\hat\beta}\right)\) empirically, without
imposing structural assumptions on \(\boldsymbol\Sigma_i\). There are
several versions of this approach, all of which can be written as
\begin{equation}
\label{eq:V_small}
\mathbf{V}^{CR} = \mathbf{M_{\ddot{R}}}\left(\sum_{i=1}^m \mathbf{\ddot{R}}_i'\mathbf{W}_i \mathbf{A}_i \mathbf{e}_i \mathbf{e}_i' \mathbf{A}_i' \mathbf{W}_i \mathbf{\ddot{R}}_i\right) \mathbf{M_{\ddot{R}}},
\end{equation} where
\(\mathbf{e}_i = \mathbf{Y}_i - \mathbf{X}_i \boldsymbol{\hat\beta}\) is
the vector of residuals from cluster \(i\) and \(\mathbf{A}_i\) is some
\(n_i\) by \(n_i\) adjustment matrix.

The form of the adjustment matrices parallels those of the
heteroskedasticity-consistent variance estimators proposed by
\citet{MacKinnon1985some}. The original CRVE, described by
\citet{Liang1986longitudinal}, uses \(\mathbf{A}_i = \mathbf{I}_i\), an
\(n_i \times n_i\) identity matrix. Following
\citet{Cameron2015practitioners}, we refer to this estimator as CR0.
This estimator is biased towards zero because the cross-product of the
residuals \(\mathbf{e}_i \mathbf{e}_i'\) tends to under-estimate the
true variance \(\boldsymbol\Sigma_i\) in cluster \(i\). A rough bias
adjustment is to take \(\mathbf{A}_i = c\mathbf{I}_i\), where
\(c = \sqrt{(m/(m-1))}\); we denote this adjusted estimator as CR1. Some
functions in Stata use a slightly different correction factor
\(c_S = \sqrt{(m N)/[(m - 1)(N - p)]}\); we will refer to the adjusted
estimator using \(c_S\) as CR1S. When \(N >> p\),
\(c_S \approx \sqrt{m/(m-1)}\) and so CR1 and CR1S will be very similar.
However, CR1 and CR1S can differ quite substantially for models with a
large number of fixed effects and small within-cluster sample size;
recent guidance emphasizes that CR1S is not appropriate for this
scenario \citet{Cameron2015practitioners}. The CR1 and CR1S estimators
are commonly used in empirical applications.

Use of these adjustments still tends to under-estimate the true variance
of \(\hat{\boldsymbol\beta}\) because the degree of bias depends not
only on the number of clusters \(m\), but also on skewness of the
covariates and unbalance across clusters
\citep{Carter2013asymptotic, MacKinnon2013thirty, Cameron2015practitioners, Young2016improved}.
A more principled approach to bias correction would take into account
the features of the covariates in \(\mathbf{X}\). One such estimator
uses adjustment matrices given by
\(\mathbf{A}_i = \left(\mathbf{I} - \mathbf{\ddot{R}}_i \mathbf{M_{\ddot{R}}}\mathbf{\ddot{R}}_i'\mathbf{W}_i\right)^{-1}\).
This estimator, denoted CR3, closely approximates the jackknife
re-sampling estimator \citep{Bell2002bias, Mancl2001covariance}.\\
However, CR3 tends to over-correct the bias of CR0, while the CR1
estimator tends to under-correct. The next section describes in detail
the BRL approach, which makes adjustments that are intermediate in
magnitude between CR1 and CR3.

\hypertarget{sec:BRL}{%
\section{BIAS REDUCED LINEARIZATION}\label{sec:BRL}}

The BRL correction is premised on a ``working'' model for the structure
of the errors, which must be specified by the analyst. Under a given
working model, adjustment matrices \(\mathbf{A}_i\) are defined so that
the variance estimator is exactly unbiased. We refer to this correction
as CR2 because it extends the HC2 variance estimator for regressions
with uncorrelated errors, which is exactly unbiased when the errors are
homoskedastic \citep{MacKinnon1985some}. The idea of specifying a model
may seem antithetical to the purpose of using CRVE, yet extensive
simulation studies have demonstrated that the method performs well in
small samples even when the working model is incorrect
\citep{Tipton2015small-t, Bell2002bias, Cameron2015practitioners, Imbens2015robust}.
Although the CR2 estimator might not be exactly unbiased when the
working model is misspecified, its bias still tends to be greatly
reduced compared to CR1 or CR0 (thus the name ``bias reduced
linearization''). Furthermore, as the number of clusters increases,
reliance on the working model diminishes.

Let
\(\boldsymbol\Phi = \text{diag}\left(\boldsymbol\Phi_1,...,\boldsymbol\Phi_m\right)\)
denote a working model for the covariance structure (up to a scalar
constant). For example, we might assume that the errors are uncorrelated
and homoskedastic, with \(\boldsymbol\Phi_i = \mathbf{I}_i\) for
\(i = 1,...,m\). Alternatively, \citet{Imbens2015robust} suggested using
a random effects (i.e., compound symmetric) structure, in which
\(\boldsymbol\Phi_i\) has unit diagonal entries and off-diagonal entries
of \(\rho\), with \(\rho\) estimated using the OLS residuals.

In the original formulation of \citet{Bell2002bias}, the BRL adjustment
matrices are chosen to satisfy the criterion \begin{equation}
\label{eq:CR2_criterion_BM}
\mathbf{A}_i \left(\mathbf{I} - \mathbf{H_X}\right)_i \boldsymbol\Phi \left(\mathbf{I} - \mathbf{H_X}\right)_i' \mathbf{A}_i'  =  \boldsymbol\Phi_i 
\end{equation} for a given working model, where
\(\left(\mathbf{I} - \mathbf{H_X}\right)_i\) denotes the rows of
\(\mathbf{I} - \mathbf{H_X}\) corresponding to cluster \(i\). If the
working model and weight matrices are both taken to be identity
matrices, then the adjustment matrices simplify to
\(\mathbf{A}_i = \left(\mathbf{I}_i - \mathbf{\ddot{U}}_i \mathbf{M_{\ddot{U}}} \mathbf{\ddot{U}}_i'\right)^{-1/2}\),
where \(\mathbf{Z}^{-1/2}\) denotes the symmetric square-root of the
matrix \(\mathbf{Z}\).

\hypertarget{a-more-general-brl-criterion}{%
\subsection{A more general BRL
criterion}\label{a-more-general-brl-criterion}}

The original formulation of \(\mathbf{A}_i\) is problematic because, for
some fixed effects models that are common in economic applications,
Equation \ref{eq:CR2_criterion_BM} has no solution.
\citet{Angrist2009mostly} note that this problem occurs in balanced
state-by-year panel models that include fixed effects for states and for
years, where
\(\mathbf{I}_i - \mathbf{\ddot{U}}_i \mathbf{M_{\ddot{U}}} \mathbf{\ddot{U}}_i'\)
is not of full rank. \citet{Young2016improved} reported that this
problem occurred frequently when applying BRL to a large corpus of
fitted regression models drawn from published studies.

This issue can be solved by using an alternative criterion to define the
adjustment matrices, for which a solution always exists. Instead of
(\ref{eq:CR2_criterion_BM}), we propose to use adjustment matrices
\(\mathbf{A}_i\) that satisfy: \begin{equation}
\label{eq:CR2_criterion}
\mathbf{\ddot{R}}_i' \mathbf{W}_i \mathbf{A}_i \left(\mathbf{I} - \mathbf{H_X}\right)_i \boldsymbol\Phi \left(\mathbf{I} - \mathbf{H_X}\right)_i' \mathbf{A}_i' \mathbf{W}_i \mathbf{\ddot{R}}_i = \mathbf{\ddot{R}}_i' \mathbf{W}_i \boldsymbol\Phi_i \mathbf{W}_i \mathbf{\ddot{R}}_i.
\end{equation} A variance estimator that uses such adjustment matrices
will be exactly unbiased when the working model is correctly specified.

A symmetric solution to Equation (\ref{eq:CR2_criterion}) is given by
\begin{equation}
\label{eq:CR2_adjustment}
\mathbf{A}_i = \mathbf{D}_i' \mathbf{B}_i^{+1/2} \mathbf{D}_i,
\end{equation} where \(\mathbf{D}_i\) is the upper-right triangular
Cholesky factorization of \(\boldsymbol\Phi_i\), \begin{equation}
\label{eq:CR2_Bmatrix}
\mathbf{B}_i = \mathbf{D}_i\left(\mathbf{I} - \mathbf{H_X}\right)_i \boldsymbol\Phi \left(\mathbf{I} - \mathbf{H_X}\right)_i' \mathbf{D}_i',
\end{equation} and \(\mathbf{B}_i^{+1/2}\) is the symmetric square root
of the Moore-Penrose inverse of \(\mathbf{B}_i\). The Moore-Penrose
inverse of \(\mathbf{B}_i\) is well-defined and unique \citep[Thm.
9.18]{Banerjee2014linear}. In contrast, the original BRL adjustment
matrices involve the symmetric square root of the regular inverse of
\(\mathbf{B}_i\), which does not exist when \(\mathbf{B}_i\) is
rank-deficient. If \(\mathbf{B}_i\) is of full rank, then our adjustment
matrices reduce to the original formulation described by
\citet{Bell2002bias}.

The adjustment matrices given by (\ref{eq:CR2_adjustment}) and
(\ref{eq:CR2_Bmatrix}) satisfy criterion (\ref{eq:CR2_criterion}), as
stated in the following theorem.

\begin{thm}
\label{thm:BRL_FE}
Let $\mathbf{L}_i = \left(\mathbf{\ddot{U}}'\mathbf{W}\mathbf{\ddot{U}} - \mathbf{\ddot{U}}_i'\mathbf{W}_i\mathbf{\ddot{U}}_i\right)$, where $\mathbf{\ddot{U}} = \left(\mathbf{I} - \mathbf{H_T}\right)\mathbf{U}$, and assume that $\mathbf{L}_1,...,\mathbf{L}_m$ have full rank $r + s$. Further assume that $\text{Var}\left(\boldsymbol\epsilon_i\left|\mathbf{X}_i\right.\right) = \sigma^2 \boldsymbol\Phi_i$, for $i = 1,...,m$. Then the adjustment matrix $\mathbf{A}_i$ defined in (\ref{eq:CR2_adjustment}) and (\ref{eq:CR2_Bmatrix}) satisfies criterion (\ref{eq:CR2_criterion}) and the CR2 variance estimator is exactly unbiased.
\end{thm}

Proof is given in the supplementary materials. The main implication of
Theorem \ref{thm:BRL_FE} is that, under our more general definition, the
CR2 variance estimator remains well-defined even in models with large
sets of fixed effects.

\hypertarget{absorption-and-lsdv-equivalence}{%
\subsection{Absorption and LSDV
Equivalence}\label{absorption-and-lsdv-equivalence}}

In fixed effects regression models, a problem with the original
definition of BRL is that it can result in a different estimator
depending upon which design matrix is used. If \(\boldsymbol\beta\) is
estimated using LSDV, then it is natural to calculate the CR2 adjustment
matrices based on the full covariate design matrix, \(\mathbf{X}\).
However, if \(\boldsymbol\beta\) is estimated after absorbing the fixed
effects, the analyst might choose to calculate the CR2 correction based
on the absorbed covariate matrix \(\mathbf{\ddot{R}}\)---that is, by
substituting \(\mathbf{H_{\ddot{R}}}\) for \(\mathbf{H_X}\) in
(\ref{eq:CR2_Bmatrix})---in order to avoid calculating the full
projection matrix \(\mathbf{H_X}\). This approach can lead to different
adjustment matrices because it is based on a subtly different working
model. Essentially, calculating CR2 based on \(\mathbf{H_{\ddot{R}}}\)
amounts to assuming that the working model \(\boldsymbol\Phi\) applies
not to the model errors \(\boldsymbol\epsilon\), but rather to the
errors from the final-stage regression of \(\mathbf{\ddot{y}}\) on
\(\mathbf{\ddot{R}}\). Because the CR2 adjustment is relatively
insensitive to the working model, the difference between accounting for
or ignoring absorption will in many instances be small. We investigate
the differences between the approaches as part of the simulation study
in Section \ref{sec:simulation}.

When based on the full regression model, a drawback of using the CR2
adjustment matrices is that it entails calculating the projection matrix
\(\mathbf{H_X}\) for the full set of \(p\) covariates (i.e., including
fixed effect indicators). Given that the entire advantage of using
absorption to calculate \(\hat{\boldsymbol\beta}\) is to avoid
computations involving large, sparse matrices, it is of interest to find
methods for more efficiently calculating the CR2 adjustment matrices.
Some computational efficiency can be gained by using the fact that the
residual projection matrix \(\mathbf{I} - \mathbf{H_X}\) can be factored
into components as
\(\left(\mathbf{I} - \mathbf{H_X}\right)_i = \left(\mathbf{I} - \mathbf{H_{\ddot{R}}}\right)_i \left(\mathbf{I} - \mathbf{H_{\ddot{S}}}\right) \left(\mathbf{I} - \mathbf{H_T}\right)\).

In certain circumstances, further computational efficiency can be
achieved by computing the adjustment matrices after absorbing the
within-cluster fixed effects \(\mathbf{T}\) (but not the between-cluster
fixed effects \(\mathbf{S}\)). Specifically, if the weights used for WLS
estimation are the inverses of the working covariance model, so that
\(\mathbf{W}_i = \boldsymbol\Phi_i^{-1}\) for \(i = 1,...,m\), then the
adjustment matrices can be calculated without accounting for the
within-cluster fixed effects. This result is formalized in the following
theorem.

\begin{thm}
\label{thm:absorb}
Let $\mathbf{\tilde{A}}_i = \mathbf{D}_i'\mathbf{\tilde{B}}_i^{+1/2} \mathbf{D}_i$, where 
\begin{equation}
\label{eq:CR2_B_tilde}
\mathbf{\tilde{B}}_i = \mathbf{D}_i\left(\mathbf{I} - \mathbf{H_{\ddot{R}}}\right)_i \left(\mathbf{I} - \mathbf{H_{\ddot{S}}}\right) \boldsymbol\Phi \left(\mathbf{I} - \mathbf{H_{\ddot{S}}}\right)' \left(\mathbf{I} - \mathbf{H_{\ddot{R}}}\right)_i' \mathbf{D}_i'.
\end{equation}
If $\mathbf{W} = \boldsymbol\Phi^{-1}$ and $\mathbf{T}_i \mathbf{T}_k' = \mathbf{0}$ for $i \neq k$, then $\mathbf{A}_i = \mathbf{\tilde{A}}_i$. 
\end{thm}

Proof is given in the supplementary materials. The main implication of
Theorem \ref{thm:absorb} is that the more computationally tractable
formula \(\mathbf{\tilde{B}}_i\) can be used in the common case that the
weighting matrices are the inverse of the working covariance model.
Following the working model suggested by \citet{Bell2002bias}, in which
\(\boldsymbol\Phi = \mathbf{I}\), the theorem shows that the adjustment
method is invariant to the choice of estimator so long as the model is
estimated by OLS (i.e., with \(\mathbf{W} = \mathbf{I}\)). \%In this
case, the CR2 adjustment matrices then simplify further to
\(\mathbf{A}_i = \left(\mathbf{I}_i - \mathbf{\ddot{U}}_i\left(\mathbf{\ddot{U}}'\mathbf{\ddot{U}}\right)^{-1}\mathbf{\ddot{U}}_i'\right)^{+1/2}\).
In contrast, if the working model proposed by \citet{Imbens2015robust}
is instead used (while still using OLS), then the the CR2 adjustments
might differ depending on whether LSDV or the fixed effects estimator is
used.

\newpage

\bibliographystyle{agsm}
\bibliography{bibliography.bib}


\end{document}
